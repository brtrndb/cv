% !TEX encoding = UTF-8 Unicode
% !TEX TS-program = LuaLaTeX

%% Packages.
\documentclass[10pt,sans,a4paper]{moderncv}
\usepackage{luatextra}
\usepackage[french]{babel}
\usepackage[paperheight=36.1cm]{geometry}
\usepackage[fixed]{fontawesome5}
\usepackage{wrapfig}
\usepackage{ifthen}
\usepackage{xcolor}
\usepackage{textcomp}
\usepackage{eso-pic}
\usepackage{xargs}
\usepackage{expl3}
\usepackage{multirow}
\usepackage{adjustbox}

% Margin options.
\geometry{top=1.50 cm}
\geometry{bottom=1.50 cm}
\geometry{left=1.50 cm}
\geometry{right=1.50 cm}

%% moderncv configuration options.
\moderncvcolor{blue}
\moderncvstyle{classic}
\moderncvicons{awesome}
\nopagenumbers{}
\setlength{\hintscolumnwidth}{+3.20 cm} % Column width of cvitem.

%% Page border.
\newlength{\pageborder}
\setlength{\pageborder}{3 pt}
\pagecolor{color1}
\AddToShipoutPictureBG{%
  \AtPageLowerLeft{%
    \color{white}%
    \hspace{\pageborder}%
    \rule[\pageborder]{\paperwidth-2\pageborder}{\paperheight-2\pageborder}%
  }%
}

%% Import aliases.
% !TEX encoding = UTF-8 Unicode
% !TEX TS-program = LuaLaTeX

%% Packages.
\documentclass[10pt,a4paper]{moderncv}
\usepackage{luatextra}
\usepackage[french]{babel}
\usepackage{geometry}
\usepackage[fixed]{fontawesome5}
\usepackage{wrapfig}
\usepackage{ifthen}
\usepackage{xcolor}
\usepackage{textcomp}
\usepackage{eso-pic}
\usepackage{xargs}
\usepackage{expl3}
\usepackage{multirow}
\usepackage{adjustbox}

% Margin options.
\geometry{top=1.50 cm}
\geometry{bottom=1.50 cm}
\geometry{left=1.50 cm}
\geometry{right=1.50 cm}

%% moderncv configuration options.
\moderncvcolor{blue}
\moderncvstyle{classic}
\moderncvicons{awesome}
\nopagenumbers{}
\renewcommand{\familydefault}{\sfdefault}
\setlength{\hintscolumnwidth}{+3.20 cm} % Column width of cvitem.

%% Page border.
\newlength{\pageborder}
\setlength{\pageborder}{3 pt}
\pagecolor{color1}
\AddToShipoutPictureBG{%
  \AtPageLowerLeft{%
    \color{white}%
    \hspace{\pageborder}%
    \rule[\pageborder]{\paperwidth-2\pageborder}{\paperheight-2\pageborder}%
  }%
}

%% Import aliases.
% !TEX encoding = UTF-8 Unicode
% !TEX TS-program = LuaLaTeX

%% Packages.
\documentclass[10pt,a4paper]{moderncv}
\usepackage{luatextra}
\usepackage[french]{babel}
\usepackage{geometry}
\usepackage[fixed]{fontawesome5}
\usepackage{wrapfig}
\usepackage{ifthen}
\usepackage{xcolor}
\usepackage{textcomp}
\usepackage{eso-pic}
\usepackage{xargs}
\usepackage{expl3}
\usepackage{multirow}
\usepackage{adjustbox}

% Margin options.
\geometry{top=1.50 cm}
\geometry{bottom=1.50 cm}
\geometry{left=1.50 cm}
\geometry{right=1.50 cm}

%% moderncv configuration options.
\moderncvcolor{blue}
\moderncvstyle{classic}
\moderncvicons{awesome}
\nopagenumbers{}
\renewcommand{\familydefault}{\sfdefault}
\setlength{\hintscolumnwidth}{+3.20 cm} % Column width of cvitem.

%% Page border.
\newlength{\pageborder}
\setlength{\pageborder}{3 pt}
\pagecolor{color1}
\AddToShipoutPictureBG{%
  \AtPageLowerLeft{%
    \color{white}%
    \hspace{\pageborder}%
    \rule[\pageborder]{\paperwidth-2\pageborder}{\paperheight-2\pageborder}%
  }%
}

%% Import aliases.
% !TEX encoding = UTF-8 Unicode
% !TEX TS-program = LuaLaTeX

%% Packages.
\documentclass[10pt,a4paper]{moderncv}
\usepackage{luatextra}
\usepackage[french]{babel}
\usepackage{geometry}
\usepackage[fixed]{fontawesome5}
\usepackage{wrapfig}
\usepackage{ifthen}
\usepackage{xcolor}
\usepackage{textcomp}
\usepackage{eso-pic}
\usepackage{xargs}
\usepackage{expl3}
\usepackage{multirow}
\usepackage{adjustbox}

% Margin options.
\geometry{top=1.50 cm}
\geometry{bottom=1.50 cm}
\geometry{left=1.50 cm}
\geometry{right=1.50 cm}

%% moderncv configuration options.
\moderncvcolor{blue}
\moderncvstyle{classic}
\moderncvicons{awesome}
\nopagenumbers{}
\renewcommand{\familydefault}{\sfdefault}
\setlength{\hintscolumnwidth}{+3.20 cm} % Column width of cvitem.

%% Page border.
\newlength{\pageborder}
\setlength{\pageborder}{3 pt}
\pagecolor{color1}
\AddToShipoutPictureBG{%
  \AtPageLowerLeft{%
    \color{white}%
    \hspace{\pageborder}%
    \rule[\pageborder]{\paperwidth-2\pageborder}{\paperheight-2\pageborder}%
  }%
}

%% Import aliases.
\input{./BertrandBoyer.aliases}

%% Personal information.
\name{\textbf{Bertrand}}{\textbf{Boyer}}
\title{\texorpdfstring{Java Software Engineer\newline\large{\SpringBoot{} - \RD{} - Innovation}}{Java Software Engineer}}
\address{}{}{}
\phone[mobile]{+33 (0) 6 10 04 27 16}
% \phone[fixed]{}
% \phone[fax]{}
\email{boyer.bertrand@gmail.com}
\social[linkedin]{BertrandBoyer}
\social[github]{BrtrndB}
\homepage{brtrndb.github.io}
% \social[twitter]{}
% \photo[64pt][0.5pt]{./img/profile}
% \extrainfo{\href{https://youtu.be/pq31rjX1BMw}{Epitech Promotion 2015}}
\quote{%
  \vspace{-2.50 em}\newline%
  « Basically, I transform bad coffee \faCoffee{} into good code \faKeyboard. »%
}%

\begin{document}
  %%%%%%%%%%%%%%%%%%%%%%%%%%%%%%%%%%%%%%%%%%%%%%%%%%
  %% Import custom macros for CV layout.
  \input{./BertrandBoyer.macros}

  %%%%%%%%%%%%%%%%%%%%%%%%%%%%%%%%%%%%%%%%%%%%%%%%%%
  %% Title & Personal information.

  \makecvtitle

  %%%%%%%%%%%%%%%%%%%%%%%%%%%%%%%%%%%%%%%%%%%%%%%%%%

  \vspace{-3.50 em}

  %%%%%%%%%%%%%%%%%%%%%%%%%%%%%%%%%%%%%%%%%%%%%%%%%%
  %% Skills section.
  \cvsection{\faCode}{Compétences techniques}

  \skillEntry{\faCoffee}{Écosystème \Java}{\SpringBoot, \RxJava, \Hibernate, \QueryDSL, \Flyway, \Testcontainers, \Gradle.}
  \skillEntry{\faDatabase}{BDD \& Système}{\Postgres, \MongoDB, \ElasticSearch, \Scylla, \Bash, \Docker, \Linux.}
  \skillEntry{\faGlobeAfrica}{Web \& Messaging}{\REST, \OpenAPI, \JWT, \OAuthTwo, \NATS, \PubSub, \React, \TypeScript.}
  \skillEntry{\faProjectDiagram}{Méthodologies}{\HexagonalArch, \CleanCode, \TDD, \Scrum, \Kanban, \DX.}
  \skillEntry{\faTools}{Outils}{\IntelliJ, \Git, \GitHub, \GitHubActions, \Postman, \Lens, \Grafana, \Jira, \LinkLaTeX.}

  %%%%%%%%%%%%%%%%%%%%%%%%%%%%%%%%%%%%%%%%%%%%%%%%%%

  \vspace{\cvSectionSpace}

  %%%%%%%%%%%%%%%%%%%%%%%%%%%%%%%%%%%%%%%%%%%%%%%%%%
  %% Experiences section.
  \cvsection{\faBriefcase}{Expériences professionnelles}

  \experienceEntry{\Happn}{Paris}{\remoteIcon}{Février 2024}{}{Java Software Engineer}{\#Rencontre \#App \#Legacy \faHeart}{%
    \experienceEntryContent{\happnURL}{./img/happn}
    {Développeur dans l'équipe backend d'\Happn, l'application de rencontre qui permet de retrouver les personnes croisées, que ce soit dans la rue, un café, ou les transports.}
    {
        {Développement de nouvelles features phares (\textit{\SmartLOL}, \textit{\PerfectDates}).},
        {Migration des microservices existants vers une \MakeLowercase{\HexagonalArch}.},
        {Dette \& upgrades: \Java{} 11/17$\;$\rightarrow{} 21, \SpringBoot{} 2$\;$\rightarrow{} 3, \Maven{}$\;$\rightarrow{} \Gradle.}
    }
    {\JavaTwentyOne, \SpringBoot, \ElasticSearch, \Scylla, \PubSub, \OpenAPI}
  }

  \experienceEntry{\SetKeeper}{Paris}{\remoteIcon}{Juin 2022}{Novembre 2023}{Java Software Engineer}{\#Cinéma \#Paie \#US \faFilm}{%
    \experienceEntryContent{\setkeeperURL}{./img/setkeeper}
    {Développeur backend dans l'équipe produit de \SetKeeper, une plateforme de gestion, d'organisation et de planification pour les équipes de tournage \href{\setkeeperClientsURL}{(films, séries, TV)}.}
    {
        {Mise en place d'outils et méthodes pour améliorer la \textit{Developer Experience} (\DX).},
        {Intégration de \Google{} (\GoogleDrive{} \& \GoogleContacts) pour du partage de documents générés.},
        {Migration d'une partie du code legacy vers une stack plus à jour.}
    }
    {\JavaSeventeen, \Vertx, \RxJava, \MongoDB, \Maven, \Docker, \OpenAPI}
  }

  \experienceEntry{\Metroscope}{Paris}{\remoteIcon}{Août 2019}{Juin 2022}{Full Stack Developer}{\#Energie \#Maintenance \#\IoT{} \faIndustry}{%
    \experienceEntryContent{\metroscopeURL}{./img/metroscope}
    {Développeur front puis back dans l'équipe produit de \Metroscope, un logiciel d'aide à l'analyse et la détection de défaillances sur les systèmes de production d'énergie.}
    {
        {Réécriture \textit{*from scratch*} de l'application web en \React{} et \TypeScript.},
        {Division du backend monolithique en microservices \& création de nouveaux services.},
        {Passage à une communication évènementielle entre les différents microservices.},
        {Veille et partage de connaissances via l'organisation de \textit{"Backend Chapters"}.}
    }
    {\JavaEleven, \Kotlin, \SpringBoot, \NATS, \Docker, \React, \TypeScript}
  }

  \experienceEntry{\Wemanity}{Paris}{}{Mars 2018}{Juillet 2019}{Full Stack Lead Developer}{\#Banque \#Innovation \#IA \faMoneyBill*}{%
    \experienceEntryContent{\wemanityURL}{./img/wemanity}
    {Lead Developer \RD{} au sein du Lab Innovation \textit{Innov8} de la \SG.}
    {
        {Idéation, création et maintenance d'applications autour des robots \Pepper.},
        {Développement d'un MVP de data-visualisation de graphes complexes en 3D.},
        {Étude et prototypage de services cognitifs (\OCR, \NLU, \STT{} et \Traduction).},
        {Formation et encadrement des developpeurs (stagiaires, alternants et juniors).}
    }
    {\React, \MaterialUI, \Loopback, \Python, \Docker, \GitLab}
  }

  \experienceEntry{\Dhatim}{Paris}{}{Octobre 2016}{Février 2018}{Java Developer}{\#Paie \#Comptabilité \#RH \faFileInvoiceDollar}{%
    \experienceEntryContent{\dhatimURL}{./img/dhatim}
    {Développeur backend dans l'équipe de \textit{\Conciliator}, une solution en SaaS de contrôle de \DSN{} pour les entreprises.}
    {
        {Prototypage et développement des nouvelles fonctionalités, \Scrum{} Master.},
        {Compréhension des besoins pour du support technique auprès des utilisateurs.},
        {Mise en place d'un POC de chatbot \Intercom{} pour du support client simple.}
    }
    {\JavaEight, \Dropwizard, \TestNG, \Postgres, \GitHub, \Jenkins}
  }

  \experienceEntry{\Visian}{Nanterre}{}{Mars 2015}{Octobre 2016}{\IoT{} Lead Developer}{\#\IoT{} \#\RD{} \#Innovation \faMicrochip}{%
    \experienceEntryContent{\visianURL}{./img/visian}
    {Développeur \RD{} puis Lead Developer chez \Visian, la filiale \IoT{} de \NeuronesIT.}
    {
        {Animation d'ateliers d'idéation, compréhension des besoins, veille technique.},
        {Développement de prototypes mêlant \IoT{}, mobile, cloud et data-visualisation.},
        {Encadrement et mentorat des équipes techniques (hardware \& software).}
    }
    {\LanguageCCPlusPlus, \JavaEight, \Android, \Python, \GitHub, \MicrosoftAzure}
  }

  %%%%%%%%%%%%%%%%%%%%%%%%%%%%%%%%%%%%%%%%%%%%%%%%%%

  \vspace{\cvSectionSpace}

  %%%%%%%%%%%%%%%%%%%%%%%%%%%%%%%%%%%%%%%%%%%%%%%%%%
  %% Education section.
  \cvsection{\faGraduationCap}{Parcours scolaire, stages \& alternances}

  \educationEntry{\Epitech}{2011}{2015}{\faUniversity}{Titre d'Expert en Technologie de l'Information}{Toulouse/Paris}{%
    \jobItems{
        {\companyName{\SII}: Prototypage \IoT{} combinant \href{\bitcrazeURL}{drone}, \RaspberryPi{} et divers capteurs (1 an).},
        {\companyName{\Thales}: Étude et déploiement d'\OpenStack{} dans un environnement de tests (6 mois).},
        {\companyName{\Novacom}: Développement d'un ensemble de tests unitaires et fonctionnels (1 an).}
    }
  }
  \educationEntry{\Griffith}{2013}{2014}{\faBeer}{Master I en Computing option Business \& Management}{Dublin}{}
  \educationEntry{\EISTI}{2008}{2011}{\faBrain}{Classes préparatoires Maths Sup/Spé (équivalence L2 Math/Info)}{Pau}{}
  \educationEntry{\Vauvenargues}{}{2008}{\faSchool}{Bac S options Maths \& Sciences de l'Ingénieur}{Aix-en-Provence}{}

  %%%%%%%%%%%%%%%%%%%%%%%%%%%%%%%%%%%%%%%%%%%%%%%%%%

  \vspace{\cvSectionSpace}

  %%%%%%%%%%%%%%%%%%%%%%%%%%%%%%%%%%%%%%%%%%%%%%%%%%
  %% Hobbies & Community sections.
  \begin{minipage}[t]{0.42\linewidth}
    \cvsection{\faTheaterMasks}{Centres d'intérêts}
    \extraEntry{\faPhotoVideo}{Photographie, musique, cinéma.}%
    \extraEntry{\faBook}{Lecture (thrillers, SF, fantasy, comics).}%
    \extraEntry{\faGamepad}{Lego, robotique, gaming, open source.}%
    \extraEntry{\faRunning}{\href{https://youtu.be/dQw4w9WgXcQ}{Urban fitting}.}%
  \end{minipage}
%
  \hspace{+2.00 cm}%
%
  \begin{minipage}[t]{0.42\linewidth}
    \cvsection{\faPeopleCarry}{Associatif}
    \extraEntry{\faCampground}{Scoutisme (en tant que scout et chef).}%
    \extraEntry{\faRobot}{Membre du Lab Robotique à \Epitech}%
    \extraEntry{\faCameraRetro}{Photographe du BDE de l'\EISTI.}%
    \extraEntry{\faVideo}{Responsable \textit{Cinéma \& Séries} à \AirEISTI.}%
  \end{minipage}

  %%%%%%%%%%%%%%%%%%%%%%%%%%%%%%%%%%%%%%%%%%%%%%%%%%

\end{document}


%% Personal information.
\name{\textbf{Bertrand}}{\textbf{Boyer}}
\title{\texorpdfstring{Java Software Engineer\newline\large{\SpringBoot{} - \RD{} - Innovation}}{Java Software Engineer}}
\address{}{}{}
\phone[mobile]{+33 (0) 6 10 04 27 16}
% \phone[fixed]{}
% \phone[fax]{}
\email{boyer.bertrand@gmail.com}
\social[linkedin]{BertrandBoyer}
\social[github]{BrtrndB}
\homepage{brtrndb.github.io}
% \social[twitter]{}
% \photo[64pt][0.5pt]{./img/profile}
% \extrainfo{\href{https://youtu.be/pq31rjX1BMw}{Epitech Promotion 2015}}
\quote{%
  \vspace{-2.50 em}\newline%
  « Basically, I transform bad coffee \faCoffee{} into good code \faKeyboard. »%
}%

\begin{document}
  %%%%%%%%%%%%%%%%%%%%%%%%%%%%%%%%%%%%%%%%%%%%%%%%%%
  %% Import custom macros for CV layout.
  % !TEX encoding = UTF-8 Unicode
% !TEX TS-program = LuaLaTeX

%% Packages.
\documentclass[10pt,a4paper]{moderncv}
\usepackage{luatextra}
\usepackage[french]{babel}
\usepackage{geometry}
\usepackage[fixed]{fontawesome5}
\usepackage{wrapfig}
\usepackage{ifthen}
\usepackage{xcolor}
\usepackage{textcomp}
\usepackage{eso-pic}
\usepackage{xargs}
\usepackage{expl3}
\usepackage{multirow}
\usepackage{adjustbox}

% Margin options.
\geometry{top=1.50 cm}
\geometry{bottom=1.50 cm}
\geometry{left=1.50 cm}
\geometry{right=1.50 cm}

%% moderncv configuration options.
\moderncvcolor{blue}
\moderncvstyle{classic}
\moderncvicons{awesome}
\nopagenumbers{}
\renewcommand{\familydefault}{\sfdefault}
\setlength{\hintscolumnwidth}{+3.20 cm} % Column width of cvitem.

%% Page border.
\newlength{\pageborder}
\setlength{\pageborder}{3 pt}
\pagecolor{color1}
\AddToShipoutPictureBG{%
  \AtPageLowerLeft{%
    \color{white}%
    \hspace{\pageborder}%
    \rule[\pageborder]{\paperwidth-2\pageborder}{\paperheight-2\pageborder}%
  }%
}

%% Import aliases.
\input{./BertrandBoyer.aliases}

%% Personal information.
\name{\textbf{Bertrand}}{\textbf{Boyer}}
\title{\texorpdfstring{Java Software Engineer\newline\large{\SpringBoot{} - \RD{} - Innovation}}{Java Software Engineer}}
\address{}{}{}
\phone[mobile]{+33 (0) 6 10 04 27 16}
% \phone[fixed]{}
% \phone[fax]{}
\email{boyer.bertrand@gmail.com}
\social[linkedin]{BertrandBoyer}
\social[github]{BrtrndB}
\homepage{brtrndb.github.io}
% \social[twitter]{}
% \photo[64pt][0.5pt]{./img/profile}
% \extrainfo{\href{https://youtu.be/pq31rjX1BMw}{Epitech Promotion 2015}}
\quote{%
  \vspace{-2.50 em}\newline%
  « Basically, I transform bad coffee \faCoffee{} into good code \faKeyboard. »%
}%

\begin{document}
  %%%%%%%%%%%%%%%%%%%%%%%%%%%%%%%%%%%%%%%%%%%%%%%%%%
  %% Import custom macros for CV layout.
  \input{./BertrandBoyer.macros}

  %%%%%%%%%%%%%%%%%%%%%%%%%%%%%%%%%%%%%%%%%%%%%%%%%%
  %% Title & Personal information.

  \makecvtitle

  %%%%%%%%%%%%%%%%%%%%%%%%%%%%%%%%%%%%%%%%%%%%%%%%%%

  \vspace{-3.50 em}

  %%%%%%%%%%%%%%%%%%%%%%%%%%%%%%%%%%%%%%%%%%%%%%%%%%
  %% Skills section.
  \cvsection{\faCode}{Compétences techniques}

  \skillEntry{\faCoffee}{Écosystème \Java}{\SpringBoot, \RxJava, \Hibernate, \QueryDSL, \Flyway, \Testcontainers, \Gradle.}
  \skillEntry{\faDatabase}{BDD \& Système}{\Postgres, \MongoDB, \ElasticSearch, \Scylla, \Bash, \Docker, \Linux.}
  \skillEntry{\faGlobeAfrica}{Web \& Messaging}{\REST, \OpenAPI, \JWT, \OAuthTwo, \NATS, \PubSub, \React, \TypeScript.}
  \skillEntry{\faProjectDiagram}{Méthodologies}{\HexagonalArch, \CleanCode, \TDD, \Scrum, \Kanban, \DX.}
  \skillEntry{\faTools}{Outils}{\IntelliJ, \Git, \GitHub, \GitHubActions, \Postman, \Lens, \Grafana, \Jira, \LinkLaTeX.}

  %%%%%%%%%%%%%%%%%%%%%%%%%%%%%%%%%%%%%%%%%%%%%%%%%%

  \vspace{\cvSectionSpace}

  %%%%%%%%%%%%%%%%%%%%%%%%%%%%%%%%%%%%%%%%%%%%%%%%%%
  %% Experiences section.
  \cvsection{\faBriefcase}{Expériences professionnelles}

  \experienceEntry{\Happn}{Paris}{\remoteIcon}{Février 2024}{}{Java Software Engineer}{\#Rencontre \#App \#Legacy \faHeart}{%
    \experienceEntryContent{\happnURL}{./img/happn}
    {Développeur dans l'équipe backend d'\Happn, l'application de rencontre qui permet de retrouver les personnes croisées, que ce soit dans la rue, un café, ou les transports.}
    {
        {Développement de nouvelles features phares (\textit{\SmartLOL}, \textit{\PerfectDates}).},
        {Migration des microservices existants vers une \MakeLowercase{\HexagonalArch}.},
        {Dette \& upgrades: \Java{} 11/17$\;$\rightarrow{} 21, \SpringBoot{} 2$\;$\rightarrow{} 3, \Maven{}$\;$\rightarrow{} \Gradle.}
    }
    {\JavaTwentyOne, \SpringBoot, \ElasticSearch, \Scylla, \PubSub, \OpenAPI}
  }

  \experienceEntry{\SetKeeper}{Paris}{\remoteIcon}{Juin 2022}{Novembre 2023}{Java Software Engineer}{\#Cinéma \#Paie \#US \faFilm}{%
    \experienceEntryContent{\setkeeperURL}{./img/setkeeper}
    {Développeur backend dans l'équipe produit de \SetKeeper, une plateforme de gestion, d'organisation et de planification pour les équipes de tournage \href{\setkeeperClientsURL}{(films, séries, TV)}.}
    {
        {Mise en place d'outils et méthodes pour améliorer la \textit{Developer Experience} (\DX).},
        {Intégration de \Google{} (\GoogleDrive{} \& \GoogleContacts) pour du partage de documents générés.},
        {Migration d'une partie du code legacy vers une stack plus à jour.}
    }
    {\JavaSeventeen, \Vertx, \RxJava, \MongoDB, \Maven, \Docker, \OpenAPI}
  }

  \experienceEntry{\Metroscope}{Paris}{\remoteIcon}{Août 2019}{Juin 2022}{Full Stack Developer}{\#Energie \#Maintenance \#\IoT{} \faIndustry}{%
    \experienceEntryContent{\metroscopeURL}{./img/metroscope}
    {Développeur front puis back dans l'équipe produit de \Metroscope, un logiciel d'aide à l'analyse et la détection de défaillances sur les systèmes de production d'énergie.}
    {
        {Réécriture \textit{*from scratch*} de l'application web en \React{} et \TypeScript.},
        {Division du backend monolithique en microservices \& création de nouveaux services.},
        {Passage à une communication évènementielle entre les différents microservices.},
        {Veille et partage de connaissances via l'organisation de \textit{"Backend Chapters"}.}
    }
    {\JavaEleven, \Kotlin, \SpringBoot, \NATS, \Docker, \React, \TypeScript}
  }

  \experienceEntry{\Wemanity}{Paris}{}{Mars 2018}{Juillet 2019}{Full Stack Lead Developer}{\#Banque \#Innovation \#IA \faMoneyBill*}{%
    \experienceEntryContent{\wemanityURL}{./img/wemanity}
    {Lead Developer \RD{} au sein du Lab Innovation \textit{Innov8} de la \SG.}
    {
        {Idéation, création et maintenance d'applications autour des robots \Pepper.},
        {Développement d'un MVP de data-visualisation de graphes complexes en 3D.},
        {Étude et prototypage de services cognitifs (\OCR, \NLU, \STT{} et \Traduction).},
        {Formation et encadrement des developpeurs (stagiaires, alternants et juniors).}
    }
    {\React, \MaterialUI, \Loopback, \Python, \Docker, \GitLab}
  }

  \experienceEntry{\Dhatim}{Paris}{}{Octobre 2016}{Février 2018}{Java Developer}{\#Paie \#Comptabilité \#RH \faFileInvoiceDollar}{%
    \experienceEntryContent{\dhatimURL}{./img/dhatim}
    {Développeur backend dans l'équipe de \textit{\Conciliator}, une solution en SaaS de contrôle de \DSN{} pour les entreprises.}
    {
        {Prototypage et développement des nouvelles fonctionalités, \Scrum{} Master.},
        {Compréhension des besoins pour du support technique auprès des utilisateurs.},
        {Mise en place d'un POC de chatbot \Intercom{} pour du support client simple.}
    }
    {\JavaEight, \Dropwizard, \TestNG, \Postgres, \GitHub, \Jenkins}
  }

  \experienceEntry{\Visian}{Nanterre}{}{Mars 2015}{Octobre 2016}{\IoT{} Lead Developer}{\#\IoT{} \#\RD{} \#Innovation \faMicrochip}{%
    \experienceEntryContent{\visianURL}{./img/visian}
    {Développeur \RD{} puis Lead Developer chez \Visian, la filiale \IoT{} de \NeuronesIT.}
    {
        {Animation d'ateliers d'idéation, compréhension des besoins, veille technique.},
        {Développement de prototypes mêlant \IoT{}, mobile, cloud et data-visualisation.},
        {Encadrement et mentorat des équipes techniques (hardware \& software).}
    }
    {\LanguageCCPlusPlus, \JavaEight, \Android, \Python, \GitHub, \MicrosoftAzure}
  }

  %%%%%%%%%%%%%%%%%%%%%%%%%%%%%%%%%%%%%%%%%%%%%%%%%%

  \vspace{\cvSectionSpace}

  %%%%%%%%%%%%%%%%%%%%%%%%%%%%%%%%%%%%%%%%%%%%%%%%%%
  %% Education section.
  \cvsection{\faGraduationCap}{Parcours scolaire, stages \& alternances}

  \educationEntry{\Epitech}{2011}{2015}{\faUniversity}{Titre d'Expert en Technologie de l'Information}{Toulouse/Paris}{%
    \jobItems{
        {\companyName{\SII}: Prototypage \IoT{} combinant \href{\bitcrazeURL}{drone}, \RaspberryPi{} et divers capteurs (1 an).},
        {\companyName{\Thales}: Étude et déploiement d'\OpenStack{} dans un environnement de tests (6 mois).},
        {\companyName{\Novacom}: Développement d'un ensemble de tests unitaires et fonctionnels (1 an).}
    }
  }
  \educationEntry{\Griffith}{2013}{2014}{\faBeer}{Master I en Computing option Business \& Management}{Dublin}{}
  \educationEntry{\EISTI}{2008}{2011}{\faBrain}{Classes préparatoires Maths Sup/Spé (équivalence L2 Math/Info)}{Pau}{}
  \educationEntry{\Vauvenargues}{}{2008}{\faSchool}{Bac S options Maths \& Sciences de l'Ingénieur}{Aix-en-Provence}{}

  %%%%%%%%%%%%%%%%%%%%%%%%%%%%%%%%%%%%%%%%%%%%%%%%%%

  \vspace{\cvSectionSpace}

  %%%%%%%%%%%%%%%%%%%%%%%%%%%%%%%%%%%%%%%%%%%%%%%%%%
  %% Hobbies & Community sections.
  \begin{minipage}[t]{0.42\linewidth}
    \cvsection{\faTheaterMasks}{Centres d'intérêts}
    \extraEntry{\faPhotoVideo}{Photographie, musique, cinéma.}%
    \extraEntry{\faBook}{Lecture (thrillers, SF, fantasy, comics).}%
    \extraEntry{\faGamepad}{Lego, robotique, gaming, open source.}%
    \extraEntry{\faRunning}{\href{https://youtu.be/dQw4w9WgXcQ}{Urban fitting}.}%
  \end{minipage}
%
  \hspace{+2.00 cm}%
%
  \begin{minipage}[t]{0.42\linewidth}
    \cvsection{\faPeopleCarry}{Associatif}
    \extraEntry{\faCampground}{Scoutisme (en tant que scout et chef).}%
    \extraEntry{\faRobot}{Membre du Lab Robotique à \Epitech}%
    \extraEntry{\faCameraRetro}{Photographe du BDE de l'\EISTI.}%
    \extraEntry{\faVideo}{Responsable \textit{Cinéma \& Séries} à \AirEISTI.}%
  \end{minipage}

  %%%%%%%%%%%%%%%%%%%%%%%%%%%%%%%%%%%%%%%%%%%%%%%%%%

\end{document}


  %%%%%%%%%%%%%%%%%%%%%%%%%%%%%%%%%%%%%%%%%%%%%%%%%%
  %% Title & Personal information.

  \makecvtitle

  %%%%%%%%%%%%%%%%%%%%%%%%%%%%%%%%%%%%%%%%%%%%%%%%%%

  \vspace{-3.50 em}

  %%%%%%%%%%%%%%%%%%%%%%%%%%%%%%%%%%%%%%%%%%%%%%%%%%
  %% Skills section.
  \cvsection{\faCode}{Compétences techniques}

  \skillEntry{\faCoffee}{Écosystème \Java}{\SpringBoot, \RxJava, \Hibernate, \QueryDSL, \Flyway, \Testcontainers, \Gradle.}
  \skillEntry{\faDatabase}{BDD \& Système}{\Postgres, \MongoDB, \ElasticSearch, \Scylla, \Bash, \Docker, \Linux.}
  \skillEntry{\faGlobeAfrica}{Web \& Messaging}{\REST, \OpenAPI, \JWT, \OAuthTwo, \NATS, \PubSub, \React, \TypeScript.}
  \skillEntry{\faProjectDiagram}{Méthodologies}{\HexagonalArch, \CleanCode, \TDD, \Scrum, \Kanban, \DX.}
  \skillEntry{\faTools}{Outils}{\IntelliJ, \Git, \GitHub, \GitHubActions, \Postman, \Lens, \Grafana, \Jira, \LinkLaTeX.}

  %%%%%%%%%%%%%%%%%%%%%%%%%%%%%%%%%%%%%%%%%%%%%%%%%%

  \vspace{\cvSectionSpace}

  %%%%%%%%%%%%%%%%%%%%%%%%%%%%%%%%%%%%%%%%%%%%%%%%%%
  %% Experiences section.
  \cvsection{\faBriefcase}{Expériences professionnelles}

  \experienceEntry{\Happn}{Paris}{\remoteIcon}{Février 2024}{}{Java Software Engineer}{\#Rencontre \#App \#Legacy \faHeart}{%
    \experienceEntryContent{\happnURL}{./img/happn}
    {Développeur dans l'équipe backend d'\Happn, l'application de rencontre qui permet de retrouver les personnes croisées, que ce soit dans la rue, un café, ou les transports.}
    {
        {Développement de nouvelles features phares (\textit{\SmartLOL}, \textit{\PerfectDates}).},
        {Migration des microservices existants vers une \MakeLowercase{\HexagonalArch}.},
        {Dette \& upgrades: \Java{} 11/17$\;$\rightarrow{} 21, \SpringBoot{} 2$\;$\rightarrow{} 3, \Maven{}$\;$\rightarrow{} \Gradle.}
    }
    {\JavaTwentyOne, \SpringBoot, \ElasticSearch, \Scylla, \PubSub, \OpenAPI}
  }

  \experienceEntry{\SetKeeper}{Paris}{\remoteIcon}{Juin 2022}{Novembre 2023}{Java Software Engineer}{\#Cinéma \#Paie \#US \faFilm}{%
    \experienceEntryContent{\setkeeperURL}{./img/setkeeper}
    {Développeur backend dans l'équipe produit de \SetKeeper, une plateforme de gestion, d'organisation et de planification pour les équipes de tournage \href{\setkeeperClientsURL}{(films, séries, TV)}.}
    {
        {Mise en place d'outils et méthodes pour améliorer la \textit{Developer Experience} (\DX).},
        {Intégration de \Google{} (\GoogleDrive{} \& \GoogleContacts) pour du partage de documents générés.},
        {Migration d'une partie du code legacy vers une stack plus à jour.}
    }
    {\JavaSeventeen, \Vertx, \RxJava, \MongoDB, \Maven, \Docker, \OpenAPI}
  }

  \experienceEntry{\Metroscope}{Paris}{\remoteIcon}{Août 2019}{Juin 2022}{Full Stack Developer}{\#Energie \#Maintenance \#\IoT{} \faIndustry}{%
    \experienceEntryContent{\metroscopeURL}{./img/metroscope}
    {Développeur front puis back dans l'équipe produit de \Metroscope, un logiciel d'aide à l'analyse et la détection de défaillances sur les systèmes de production d'énergie.}
    {
        {Réécriture \textit{*from scratch*} de l'application web en \React{} et \TypeScript.},
        {Division du backend monolithique en microservices \& création de nouveaux services.},
        {Passage à une communication évènementielle entre les différents microservices.},
        {Veille et partage de connaissances via l'organisation de \textit{"Backend Chapters"}.}
    }
    {\JavaEleven, \Kotlin, \SpringBoot, \NATS, \Docker, \React, \TypeScript}
  }

  \experienceEntry{\Wemanity}{Paris}{}{Mars 2018}{Juillet 2019}{Full Stack Lead Developer}{\#Banque \#Innovation \#IA \faMoneyBill*}{%
    \experienceEntryContent{\wemanityURL}{./img/wemanity}
    {Lead Developer \RD{} au sein du Lab Innovation \textit{Innov8} de la \SG.}
    {
        {Idéation, création et maintenance d'applications autour des robots \Pepper.},
        {Développement d'un MVP de data-visualisation de graphes complexes en 3D.},
        {Étude et prototypage de services cognitifs (\OCR, \NLU, \STT{} et \Traduction).},
        {Formation et encadrement des developpeurs (stagiaires, alternants et juniors).}
    }
    {\React, \MaterialUI, \Loopback, \Python, \Docker, \GitLab}
  }

  \experienceEntry{\Dhatim}{Paris}{}{Octobre 2016}{Février 2018}{Java Developer}{\#Paie \#Comptabilité \#RH \faFileInvoiceDollar}{%
    \experienceEntryContent{\dhatimURL}{./img/dhatim}
    {Développeur backend dans l'équipe de \textit{\Conciliator}, une solution en SaaS de contrôle de \DSN{} pour les entreprises.}
    {
        {Prototypage et développement des nouvelles fonctionalités, \Scrum{} Master.},
        {Compréhension des besoins pour du support technique auprès des utilisateurs.},
        {Mise en place d'un POC de chatbot \Intercom{} pour du support client simple.}
    }
    {\JavaEight, \Dropwizard, \TestNG, \Postgres, \GitHub, \Jenkins}
  }

  \experienceEntry{\Visian}{Nanterre}{}{Mars 2015}{Octobre 2016}{\IoT{} Lead Developer}{\#\IoT{} \#\RD{} \#Innovation \faMicrochip}{%
    \experienceEntryContent{\visianURL}{./img/visian}
    {Développeur \RD{} puis Lead Developer chez \Visian, la filiale \IoT{} de \NeuronesIT.}
    {
        {Animation d'ateliers d'idéation, compréhension des besoins, veille technique.},
        {Développement de prototypes mêlant \IoT{}, mobile, cloud et data-visualisation.},
        {Encadrement et mentorat des équipes techniques (hardware \& software).}
    }
    {\LanguageCCPlusPlus, \JavaEight, \Android, \Python, \GitHub, \MicrosoftAzure}
  }

  %%%%%%%%%%%%%%%%%%%%%%%%%%%%%%%%%%%%%%%%%%%%%%%%%%

  \vspace{\cvSectionSpace}

  %%%%%%%%%%%%%%%%%%%%%%%%%%%%%%%%%%%%%%%%%%%%%%%%%%
  %% Education section.
  \cvsection{\faGraduationCap}{Parcours scolaire, stages \& alternances}

  \educationEntry{\Epitech}{2011}{2015}{\faUniversity}{Titre d'Expert en Technologie de l'Information}{Toulouse/Paris}{%
    \jobItems{
        {\companyName{\SII}: Prototypage \IoT{} combinant \href{\bitcrazeURL}{drone}, \RaspberryPi{} et divers capteurs (1 an).},
        {\companyName{\Thales}: Étude et déploiement d'\OpenStack{} dans un environnement de tests (6 mois).},
        {\companyName{\Novacom}: Développement d'un ensemble de tests unitaires et fonctionnels (1 an).}
    }
  }
  \educationEntry{\Griffith}{2013}{2014}{\faBeer}{Master I en Computing option Business \& Management}{Dublin}{}
  \educationEntry{\EISTI}{2008}{2011}{\faBrain}{Classes préparatoires Maths Sup/Spé (équivalence L2 Math/Info)}{Pau}{}
  \educationEntry{\Vauvenargues}{}{2008}{\faSchool}{Bac S options Maths \& Sciences de l'Ingénieur}{Aix-en-Provence}{}

  %%%%%%%%%%%%%%%%%%%%%%%%%%%%%%%%%%%%%%%%%%%%%%%%%%

  \vspace{\cvSectionSpace}

  %%%%%%%%%%%%%%%%%%%%%%%%%%%%%%%%%%%%%%%%%%%%%%%%%%
  %% Hobbies & Community sections.
  \begin{minipage}[t]{0.42\linewidth}
    \cvsection{\faTheaterMasks}{Centres d'intérêts}
    \extraEntry{\faPhotoVideo}{Photographie, musique, cinéma.}%
    \extraEntry{\faBook}{Lecture (thrillers, SF, fantasy, comics).}%
    \extraEntry{\faGamepad}{Lego, robotique, gaming, open source.}%
    \extraEntry{\faRunning}{\href{https://youtu.be/dQw4w9WgXcQ}{Urban fitting}.}%
  \end{minipage}
%
  \hspace{+2.00 cm}%
%
  \begin{minipage}[t]{0.42\linewidth}
    \cvsection{\faPeopleCarry}{Associatif}
    \extraEntry{\faCampground}{Scoutisme (en tant que scout et chef).}%
    \extraEntry{\faRobot}{Membre du Lab Robotique à \Epitech}%
    \extraEntry{\faCameraRetro}{Photographe du BDE de l'\EISTI.}%
    \extraEntry{\faVideo}{Responsable \textit{Cinéma \& Séries} à \AirEISTI.}%
  \end{minipage}

  %%%%%%%%%%%%%%%%%%%%%%%%%%%%%%%%%%%%%%%%%%%%%%%%%%

\end{document}


%% Personal information.
\name{\textbf{Bertrand}}{\textbf{Boyer}}
\title{\texorpdfstring{Java Software Engineer\newline\large{\SpringBoot{} - \RD{} - Innovation}}{Java Software Engineer}}
\address{}{}{}
\phone[mobile]{+33 (0) 6 10 04 27 16}
% \phone[fixed]{}
% \phone[fax]{}
\email{boyer.bertrand@gmail.com}
\social[linkedin]{BertrandBoyer}
\social[github]{BrtrndB}
\homepage{brtrndb.github.io}
% \social[twitter]{}
% \photo[64pt][0.5pt]{./img/profile}
% \extrainfo{\href{https://youtu.be/pq31rjX1BMw}{Epitech Promotion 2015}}
\quote{%
  \vspace{-2.50 em}\newline%
  « Basically, I transform bad coffee \faCoffee{} into good code \faKeyboard. »%
}%

\begin{document}
  %%%%%%%%%%%%%%%%%%%%%%%%%%%%%%%%%%%%%%%%%%%%%%%%%%
  %% Import custom macros for CV layout.
  % !TEX encoding = UTF-8 Unicode
% !TEX TS-program = LuaLaTeX

%% Packages.
\documentclass[10pt,a4paper]{moderncv}
\usepackage{luatextra}
\usepackage[french]{babel}
\usepackage{geometry}
\usepackage[fixed]{fontawesome5}
\usepackage{wrapfig}
\usepackage{ifthen}
\usepackage{xcolor}
\usepackage{textcomp}
\usepackage{eso-pic}
\usepackage{xargs}
\usepackage{expl3}
\usepackage{multirow}
\usepackage{adjustbox}

% Margin options.
\geometry{top=1.50 cm}
\geometry{bottom=1.50 cm}
\geometry{left=1.50 cm}
\geometry{right=1.50 cm}

%% moderncv configuration options.
\moderncvcolor{blue}
\moderncvstyle{classic}
\moderncvicons{awesome}
\nopagenumbers{}
\renewcommand{\familydefault}{\sfdefault}
\setlength{\hintscolumnwidth}{+3.20 cm} % Column width of cvitem.

%% Page border.
\newlength{\pageborder}
\setlength{\pageborder}{3 pt}
\pagecolor{color1}
\AddToShipoutPictureBG{%
  \AtPageLowerLeft{%
    \color{white}%
    \hspace{\pageborder}%
    \rule[\pageborder]{\paperwidth-2\pageborder}{\paperheight-2\pageborder}%
  }%
}

%% Import aliases.
% !TEX encoding = UTF-8 Unicode
% !TEX TS-program = LuaLaTeX

%% Packages.
\documentclass[10pt,a4paper]{moderncv}
\usepackage{luatextra}
\usepackage[french]{babel}
\usepackage{geometry}
\usepackage[fixed]{fontawesome5}
\usepackage{wrapfig}
\usepackage{ifthen}
\usepackage{xcolor}
\usepackage{textcomp}
\usepackage{eso-pic}
\usepackage{xargs}
\usepackage{expl3}
\usepackage{multirow}
\usepackage{adjustbox}

% Margin options.
\geometry{top=1.50 cm}
\geometry{bottom=1.50 cm}
\geometry{left=1.50 cm}
\geometry{right=1.50 cm}

%% moderncv configuration options.
\moderncvcolor{blue}
\moderncvstyle{classic}
\moderncvicons{awesome}
\nopagenumbers{}
\renewcommand{\familydefault}{\sfdefault}
\setlength{\hintscolumnwidth}{+3.20 cm} % Column width of cvitem.

%% Page border.
\newlength{\pageborder}
\setlength{\pageborder}{3 pt}
\pagecolor{color1}
\AddToShipoutPictureBG{%
  \AtPageLowerLeft{%
    \color{white}%
    \hspace{\pageborder}%
    \rule[\pageborder]{\paperwidth-2\pageborder}{\paperheight-2\pageborder}%
  }%
}

%% Import aliases.
\input{./BertrandBoyer.aliases}

%% Personal information.
\name{\textbf{Bertrand}}{\textbf{Boyer}}
\title{\texorpdfstring{Java Software Engineer\newline\large{\SpringBoot{} - \RD{} - Innovation}}{Java Software Engineer}}
\address{}{}{}
\phone[mobile]{+33 (0) 6 10 04 27 16}
% \phone[fixed]{}
% \phone[fax]{}
\email{boyer.bertrand@gmail.com}
\social[linkedin]{BertrandBoyer}
\social[github]{BrtrndB}
\homepage{brtrndb.github.io}
% \social[twitter]{}
% \photo[64pt][0.5pt]{./img/profile}
% \extrainfo{\href{https://youtu.be/pq31rjX1BMw}{Epitech Promotion 2015}}
\quote{%
  \vspace{-2.50 em}\newline%
  « Basically, I transform bad coffee \faCoffee{} into good code \faKeyboard. »%
}%

\begin{document}
  %%%%%%%%%%%%%%%%%%%%%%%%%%%%%%%%%%%%%%%%%%%%%%%%%%
  %% Import custom macros for CV layout.
  \input{./BertrandBoyer.macros}

  %%%%%%%%%%%%%%%%%%%%%%%%%%%%%%%%%%%%%%%%%%%%%%%%%%
  %% Title & Personal information.

  \makecvtitle

  %%%%%%%%%%%%%%%%%%%%%%%%%%%%%%%%%%%%%%%%%%%%%%%%%%

  \vspace{-3.50 em}

  %%%%%%%%%%%%%%%%%%%%%%%%%%%%%%%%%%%%%%%%%%%%%%%%%%
  %% Skills section.
  \cvsection{\faCode}{Compétences techniques}

  \skillEntry{\faCoffee}{Écosystème \Java}{\SpringBoot, \RxJava, \Hibernate, \QueryDSL, \Flyway, \Testcontainers, \Gradle.}
  \skillEntry{\faDatabase}{BDD \& Système}{\Postgres, \MongoDB, \ElasticSearch, \Scylla, \Bash, \Docker, \Linux.}
  \skillEntry{\faGlobeAfrica}{Web \& Messaging}{\REST, \OpenAPI, \JWT, \OAuthTwo, \NATS, \PubSub, \React, \TypeScript.}
  \skillEntry{\faProjectDiagram}{Méthodologies}{\HexagonalArch, \CleanCode, \TDD, \Scrum, \Kanban, \DX.}
  \skillEntry{\faTools}{Outils}{\IntelliJ, \Git, \GitHub, \GitHubActions, \Postman, \Lens, \Grafana, \Jira, \LinkLaTeX.}

  %%%%%%%%%%%%%%%%%%%%%%%%%%%%%%%%%%%%%%%%%%%%%%%%%%

  \vspace{\cvSectionSpace}

  %%%%%%%%%%%%%%%%%%%%%%%%%%%%%%%%%%%%%%%%%%%%%%%%%%
  %% Experiences section.
  \cvsection{\faBriefcase}{Expériences professionnelles}

  \experienceEntry{\Happn}{Paris}{\remoteIcon}{Février 2024}{}{Java Software Engineer}{\#Rencontre \#App \#Legacy \faHeart}{%
    \experienceEntryContent{\happnURL}{./img/happn}
    {Développeur dans l'équipe backend d'\Happn, l'application de rencontre qui permet de retrouver les personnes croisées, que ce soit dans la rue, un café, ou les transports.}
    {
        {Développement de nouvelles features phares (\textit{\SmartLOL}, \textit{\PerfectDates}).},
        {Migration des microservices existants vers une \MakeLowercase{\HexagonalArch}.},
        {Dette \& upgrades: \Java{} 11/17$\;$\rightarrow{} 21, \SpringBoot{} 2$\;$\rightarrow{} 3, \Maven{}$\;$\rightarrow{} \Gradle.}
    }
    {\JavaTwentyOne, \SpringBoot, \ElasticSearch, \Scylla, \PubSub, \OpenAPI}
  }

  \experienceEntry{\SetKeeper}{Paris}{\remoteIcon}{Juin 2022}{Novembre 2023}{Java Software Engineer}{\#Cinéma \#Paie \#US \faFilm}{%
    \experienceEntryContent{\setkeeperURL}{./img/setkeeper}
    {Développeur backend dans l'équipe produit de \SetKeeper, une plateforme de gestion, d'organisation et de planification pour les équipes de tournage \href{\setkeeperClientsURL}{(films, séries, TV)}.}
    {
        {Mise en place d'outils et méthodes pour améliorer la \textit{Developer Experience} (\DX).},
        {Intégration de \Google{} (\GoogleDrive{} \& \GoogleContacts) pour du partage de documents générés.},
        {Migration d'une partie du code legacy vers une stack plus à jour.}
    }
    {\JavaSeventeen, \Vertx, \RxJava, \MongoDB, \Maven, \Docker, \OpenAPI}
  }

  \experienceEntry{\Metroscope}{Paris}{\remoteIcon}{Août 2019}{Juin 2022}{Full Stack Developer}{\#Energie \#Maintenance \#\IoT{} \faIndustry}{%
    \experienceEntryContent{\metroscopeURL}{./img/metroscope}
    {Développeur front puis back dans l'équipe produit de \Metroscope, un logiciel d'aide à l'analyse et la détection de défaillances sur les systèmes de production d'énergie.}
    {
        {Réécriture \textit{*from scratch*} de l'application web en \React{} et \TypeScript.},
        {Division du backend monolithique en microservices \& création de nouveaux services.},
        {Passage à une communication évènementielle entre les différents microservices.},
        {Veille et partage de connaissances via l'organisation de \textit{"Backend Chapters"}.}
    }
    {\JavaEleven, \Kotlin, \SpringBoot, \NATS, \Docker, \React, \TypeScript}
  }

  \experienceEntry{\Wemanity}{Paris}{}{Mars 2018}{Juillet 2019}{Full Stack Lead Developer}{\#Banque \#Innovation \#IA \faMoneyBill*}{%
    \experienceEntryContent{\wemanityURL}{./img/wemanity}
    {Lead Developer \RD{} au sein du Lab Innovation \textit{Innov8} de la \SG.}
    {
        {Idéation, création et maintenance d'applications autour des robots \Pepper.},
        {Développement d'un MVP de data-visualisation de graphes complexes en 3D.},
        {Étude et prototypage de services cognitifs (\OCR, \NLU, \STT{} et \Traduction).},
        {Formation et encadrement des developpeurs (stagiaires, alternants et juniors).}
    }
    {\React, \MaterialUI, \Loopback, \Python, \Docker, \GitLab}
  }

  \experienceEntry{\Dhatim}{Paris}{}{Octobre 2016}{Février 2018}{Java Developer}{\#Paie \#Comptabilité \#RH \faFileInvoiceDollar}{%
    \experienceEntryContent{\dhatimURL}{./img/dhatim}
    {Développeur backend dans l'équipe de \textit{\Conciliator}, une solution en SaaS de contrôle de \DSN{} pour les entreprises.}
    {
        {Prototypage et développement des nouvelles fonctionalités, \Scrum{} Master.},
        {Compréhension des besoins pour du support technique auprès des utilisateurs.},
        {Mise en place d'un POC de chatbot \Intercom{} pour du support client simple.}
    }
    {\JavaEight, \Dropwizard, \TestNG, \Postgres, \GitHub, \Jenkins}
  }

  \experienceEntry{\Visian}{Nanterre}{}{Mars 2015}{Octobre 2016}{\IoT{} Lead Developer}{\#\IoT{} \#\RD{} \#Innovation \faMicrochip}{%
    \experienceEntryContent{\visianURL}{./img/visian}
    {Développeur \RD{} puis Lead Developer chez \Visian, la filiale \IoT{} de \NeuronesIT.}
    {
        {Animation d'ateliers d'idéation, compréhension des besoins, veille technique.},
        {Développement de prototypes mêlant \IoT{}, mobile, cloud et data-visualisation.},
        {Encadrement et mentorat des équipes techniques (hardware \& software).}
    }
    {\LanguageCCPlusPlus, \JavaEight, \Android, \Python, \GitHub, \MicrosoftAzure}
  }

  %%%%%%%%%%%%%%%%%%%%%%%%%%%%%%%%%%%%%%%%%%%%%%%%%%

  \vspace{\cvSectionSpace}

  %%%%%%%%%%%%%%%%%%%%%%%%%%%%%%%%%%%%%%%%%%%%%%%%%%
  %% Education section.
  \cvsection{\faGraduationCap}{Parcours scolaire, stages \& alternances}

  \educationEntry{\Epitech}{2011}{2015}{\faUniversity}{Titre d'Expert en Technologie de l'Information}{Toulouse/Paris}{%
    \jobItems{
        {\companyName{\SII}: Prototypage \IoT{} combinant \href{\bitcrazeURL}{drone}, \RaspberryPi{} et divers capteurs (1 an).},
        {\companyName{\Thales}: Étude et déploiement d'\OpenStack{} dans un environnement de tests (6 mois).},
        {\companyName{\Novacom}: Développement d'un ensemble de tests unitaires et fonctionnels (1 an).}
    }
  }
  \educationEntry{\Griffith}{2013}{2014}{\faBeer}{Master I en Computing option Business \& Management}{Dublin}{}
  \educationEntry{\EISTI}{2008}{2011}{\faBrain}{Classes préparatoires Maths Sup/Spé (équivalence L2 Math/Info)}{Pau}{}
  \educationEntry{\Vauvenargues}{}{2008}{\faSchool}{Bac S options Maths \& Sciences de l'Ingénieur}{Aix-en-Provence}{}

  %%%%%%%%%%%%%%%%%%%%%%%%%%%%%%%%%%%%%%%%%%%%%%%%%%

  \vspace{\cvSectionSpace}

  %%%%%%%%%%%%%%%%%%%%%%%%%%%%%%%%%%%%%%%%%%%%%%%%%%
  %% Hobbies & Community sections.
  \begin{minipage}[t]{0.42\linewidth}
    \cvsection{\faTheaterMasks}{Centres d'intérêts}
    \extraEntry{\faPhotoVideo}{Photographie, musique, cinéma.}%
    \extraEntry{\faBook}{Lecture (thrillers, SF, fantasy, comics).}%
    \extraEntry{\faGamepad}{Lego, robotique, gaming, open source.}%
    \extraEntry{\faRunning}{\href{https://youtu.be/dQw4w9WgXcQ}{Urban fitting}.}%
  \end{minipage}
%
  \hspace{+2.00 cm}%
%
  \begin{minipage}[t]{0.42\linewidth}
    \cvsection{\faPeopleCarry}{Associatif}
    \extraEntry{\faCampground}{Scoutisme (en tant que scout et chef).}%
    \extraEntry{\faRobot}{Membre du Lab Robotique à \Epitech}%
    \extraEntry{\faCameraRetro}{Photographe du BDE de l'\EISTI.}%
    \extraEntry{\faVideo}{Responsable \textit{Cinéma \& Séries} à \AirEISTI.}%
  \end{minipage}

  %%%%%%%%%%%%%%%%%%%%%%%%%%%%%%%%%%%%%%%%%%%%%%%%%%

\end{document}


%% Personal information.
\name{\textbf{Bertrand}}{\textbf{Boyer}}
\title{\texorpdfstring{Java Software Engineer\newline\large{\SpringBoot{} - \RD{} - Innovation}}{Java Software Engineer}}
\address{}{}{}
\phone[mobile]{+33 (0) 6 10 04 27 16}
% \phone[fixed]{}
% \phone[fax]{}
\email{boyer.bertrand@gmail.com}
\social[linkedin]{BertrandBoyer}
\social[github]{BrtrndB}
\homepage{brtrndb.github.io}
% \social[twitter]{}
% \photo[64pt][0.5pt]{./img/profile}
% \extrainfo{\href{https://youtu.be/pq31rjX1BMw}{Epitech Promotion 2015}}
\quote{%
  \vspace{-2.50 em}\newline%
  « Basically, I transform bad coffee \faCoffee{} into good code \faKeyboard. »%
}%

\begin{document}
  %%%%%%%%%%%%%%%%%%%%%%%%%%%%%%%%%%%%%%%%%%%%%%%%%%
  %% Import custom macros for CV layout.
  % !TEX encoding = UTF-8 Unicode
% !TEX TS-program = LuaLaTeX

%% Packages.
\documentclass[10pt,a4paper]{moderncv}
\usepackage{luatextra}
\usepackage[french]{babel}
\usepackage{geometry}
\usepackage[fixed]{fontawesome5}
\usepackage{wrapfig}
\usepackage{ifthen}
\usepackage{xcolor}
\usepackage{textcomp}
\usepackage{eso-pic}
\usepackage{xargs}
\usepackage{expl3}
\usepackage{multirow}
\usepackage{adjustbox}

% Margin options.
\geometry{top=1.50 cm}
\geometry{bottom=1.50 cm}
\geometry{left=1.50 cm}
\geometry{right=1.50 cm}

%% moderncv configuration options.
\moderncvcolor{blue}
\moderncvstyle{classic}
\moderncvicons{awesome}
\nopagenumbers{}
\renewcommand{\familydefault}{\sfdefault}
\setlength{\hintscolumnwidth}{+3.20 cm} % Column width of cvitem.

%% Page border.
\newlength{\pageborder}
\setlength{\pageborder}{3 pt}
\pagecolor{color1}
\AddToShipoutPictureBG{%
  \AtPageLowerLeft{%
    \color{white}%
    \hspace{\pageborder}%
    \rule[\pageborder]{\paperwidth-2\pageborder}{\paperheight-2\pageborder}%
  }%
}

%% Import aliases.
\input{./BertrandBoyer.aliases}

%% Personal information.
\name{\textbf{Bertrand}}{\textbf{Boyer}}
\title{\texorpdfstring{Java Software Engineer\newline\large{\SpringBoot{} - \RD{} - Innovation}}{Java Software Engineer}}
\address{}{}{}
\phone[mobile]{+33 (0) 6 10 04 27 16}
% \phone[fixed]{}
% \phone[fax]{}
\email{boyer.bertrand@gmail.com}
\social[linkedin]{BertrandBoyer}
\social[github]{BrtrndB}
\homepage{brtrndb.github.io}
% \social[twitter]{}
% \photo[64pt][0.5pt]{./img/profile}
% \extrainfo{\href{https://youtu.be/pq31rjX1BMw}{Epitech Promotion 2015}}
\quote{%
  \vspace{-2.50 em}\newline%
  « Basically, I transform bad coffee \faCoffee{} into good code \faKeyboard. »%
}%

\begin{document}
  %%%%%%%%%%%%%%%%%%%%%%%%%%%%%%%%%%%%%%%%%%%%%%%%%%
  %% Import custom macros for CV layout.
  \input{./BertrandBoyer.macros}

  %%%%%%%%%%%%%%%%%%%%%%%%%%%%%%%%%%%%%%%%%%%%%%%%%%
  %% Title & Personal information.

  \makecvtitle

  %%%%%%%%%%%%%%%%%%%%%%%%%%%%%%%%%%%%%%%%%%%%%%%%%%

  \vspace{-3.50 em}

  %%%%%%%%%%%%%%%%%%%%%%%%%%%%%%%%%%%%%%%%%%%%%%%%%%
  %% Skills section.
  \cvsection{\faCode}{Compétences techniques}

  \skillEntry{\faCoffee}{Écosystème \Java}{\SpringBoot, \RxJava, \Hibernate, \QueryDSL, \Flyway, \Testcontainers, \Gradle.}
  \skillEntry{\faDatabase}{BDD \& Système}{\Postgres, \MongoDB, \ElasticSearch, \Scylla, \Bash, \Docker, \Linux.}
  \skillEntry{\faGlobeAfrica}{Web \& Messaging}{\REST, \OpenAPI, \JWT, \OAuthTwo, \NATS, \PubSub, \React, \TypeScript.}
  \skillEntry{\faProjectDiagram}{Méthodologies}{\HexagonalArch, \CleanCode, \TDD, \Scrum, \Kanban, \DX.}
  \skillEntry{\faTools}{Outils}{\IntelliJ, \Git, \GitHub, \GitHubActions, \Postman, \Lens, \Grafana, \Jira, \LinkLaTeX.}

  %%%%%%%%%%%%%%%%%%%%%%%%%%%%%%%%%%%%%%%%%%%%%%%%%%

  \vspace{\cvSectionSpace}

  %%%%%%%%%%%%%%%%%%%%%%%%%%%%%%%%%%%%%%%%%%%%%%%%%%
  %% Experiences section.
  \cvsection{\faBriefcase}{Expériences professionnelles}

  \experienceEntry{\Happn}{Paris}{\remoteIcon}{Février 2024}{}{Java Software Engineer}{\#Rencontre \#App \#Legacy \faHeart}{%
    \experienceEntryContent{\happnURL}{./img/happn}
    {Développeur dans l'équipe backend d'\Happn, l'application de rencontre qui permet de retrouver les personnes croisées, que ce soit dans la rue, un café, ou les transports.}
    {
        {Développement de nouvelles features phares (\textit{\SmartLOL}, \textit{\PerfectDates}).},
        {Migration des microservices existants vers une \MakeLowercase{\HexagonalArch}.},
        {Dette \& upgrades: \Java{} 11/17$\;$\rightarrow{} 21, \SpringBoot{} 2$\;$\rightarrow{} 3, \Maven{}$\;$\rightarrow{} \Gradle.}
    }
    {\JavaTwentyOne, \SpringBoot, \ElasticSearch, \Scylla, \PubSub, \OpenAPI}
  }

  \experienceEntry{\SetKeeper}{Paris}{\remoteIcon}{Juin 2022}{Novembre 2023}{Java Software Engineer}{\#Cinéma \#Paie \#US \faFilm}{%
    \experienceEntryContent{\setkeeperURL}{./img/setkeeper}
    {Développeur backend dans l'équipe produit de \SetKeeper, une plateforme de gestion, d'organisation et de planification pour les équipes de tournage \href{\setkeeperClientsURL}{(films, séries, TV)}.}
    {
        {Mise en place d'outils et méthodes pour améliorer la \textit{Developer Experience} (\DX).},
        {Intégration de \Google{} (\GoogleDrive{} \& \GoogleContacts) pour du partage de documents générés.},
        {Migration d'une partie du code legacy vers une stack plus à jour.}
    }
    {\JavaSeventeen, \Vertx, \RxJava, \MongoDB, \Maven, \Docker, \OpenAPI}
  }

  \experienceEntry{\Metroscope}{Paris}{\remoteIcon}{Août 2019}{Juin 2022}{Full Stack Developer}{\#Energie \#Maintenance \#\IoT{} \faIndustry}{%
    \experienceEntryContent{\metroscopeURL}{./img/metroscope}
    {Développeur front puis back dans l'équipe produit de \Metroscope, un logiciel d'aide à l'analyse et la détection de défaillances sur les systèmes de production d'énergie.}
    {
        {Réécriture \textit{*from scratch*} de l'application web en \React{} et \TypeScript.},
        {Division du backend monolithique en microservices \& création de nouveaux services.},
        {Passage à une communication évènementielle entre les différents microservices.},
        {Veille et partage de connaissances via l'organisation de \textit{"Backend Chapters"}.}
    }
    {\JavaEleven, \Kotlin, \SpringBoot, \NATS, \Docker, \React, \TypeScript}
  }

  \experienceEntry{\Wemanity}{Paris}{}{Mars 2018}{Juillet 2019}{Full Stack Lead Developer}{\#Banque \#Innovation \#IA \faMoneyBill*}{%
    \experienceEntryContent{\wemanityURL}{./img/wemanity}
    {Lead Developer \RD{} au sein du Lab Innovation \textit{Innov8} de la \SG.}
    {
        {Idéation, création et maintenance d'applications autour des robots \Pepper.},
        {Développement d'un MVP de data-visualisation de graphes complexes en 3D.},
        {Étude et prototypage de services cognitifs (\OCR, \NLU, \STT{} et \Traduction).},
        {Formation et encadrement des developpeurs (stagiaires, alternants et juniors).}
    }
    {\React, \MaterialUI, \Loopback, \Python, \Docker, \GitLab}
  }

  \experienceEntry{\Dhatim}{Paris}{}{Octobre 2016}{Février 2018}{Java Developer}{\#Paie \#Comptabilité \#RH \faFileInvoiceDollar}{%
    \experienceEntryContent{\dhatimURL}{./img/dhatim}
    {Développeur backend dans l'équipe de \textit{\Conciliator}, une solution en SaaS de contrôle de \DSN{} pour les entreprises.}
    {
        {Prototypage et développement des nouvelles fonctionalités, \Scrum{} Master.},
        {Compréhension des besoins pour du support technique auprès des utilisateurs.},
        {Mise en place d'un POC de chatbot \Intercom{} pour du support client simple.}
    }
    {\JavaEight, \Dropwizard, \TestNG, \Postgres, \GitHub, \Jenkins}
  }

  \experienceEntry{\Visian}{Nanterre}{}{Mars 2015}{Octobre 2016}{\IoT{} Lead Developer}{\#\IoT{} \#\RD{} \#Innovation \faMicrochip}{%
    \experienceEntryContent{\visianURL}{./img/visian}
    {Développeur \RD{} puis Lead Developer chez \Visian, la filiale \IoT{} de \NeuronesIT.}
    {
        {Animation d'ateliers d'idéation, compréhension des besoins, veille technique.},
        {Développement de prototypes mêlant \IoT{}, mobile, cloud et data-visualisation.},
        {Encadrement et mentorat des équipes techniques (hardware \& software).}
    }
    {\LanguageCCPlusPlus, \JavaEight, \Android, \Python, \GitHub, \MicrosoftAzure}
  }

  %%%%%%%%%%%%%%%%%%%%%%%%%%%%%%%%%%%%%%%%%%%%%%%%%%

  \vspace{\cvSectionSpace}

  %%%%%%%%%%%%%%%%%%%%%%%%%%%%%%%%%%%%%%%%%%%%%%%%%%
  %% Education section.
  \cvsection{\faGraduationCap}{Parcours scolaire, stages \& alternances}

  \educationEntry{\Epitech}{2011}{2015}{\faUniversity}{Titre d'Expert en Technologie de l'Information}{Toulouse/Paris}{%
    \jobItems{
        {\companyName{\SII}: Prototypage \IoT{} combinant \href{\bitcrazeURL}{drone}, \RaspberryPi{} et divers capteurs (1 an).},
        {\companyName{\Thales}: Étude et déploiement d'\OpenStack{} dans un environnement de tests (6 mois).},
        {\companyName{\Novacom}: Développement d'un ensemble de tests unitaires et fonctionnels (1 an).}
    }
  }
  \educationEntry{\Griffith}{2013}{2014}{\faBeer}{Master I en Computing option Business \& Management}{Dublin}{}
  \educationEntry{\EISTI}{2008}{2011}{\faBrain}{Classes préparatoires Maths Sup/Spé (équivalence L2 Math/Info)}{Pau}{}
  \educationEntry{\Vauvenargues}{}{2008}{\faSchool}{Bac S options Maths \& Sciences de l'Ingénieur}{Aix-en-Provence}{}

  %%%%%%%%%%%%%%%%%%%%%%%%%%%%%%%%%%%%%%%%%%%%%%%%%%

  \vspace{\cvSectionSpace}

  %%%%%%%%%%%%%%%%%%%%%%%%%%%%%%%%%%%%%%%%%%%%%%%%%%
  %% Hobbies & Community sections.
  \begin{minipage}[t]{0.42\linewidth}
    \cvsection{\faTheaterMasks}{Centres d'intérêts}
    \extraEntry{\faPhotoVideo}{Photographie, musique, cinéma.}%
    \extraEntry{\faBook}{Lecture (thrillers, SF, fantasy, comics).}%
    \extraEntry{\faGamepad}{Lego, robotique, gaming, open source.}%
    \extraEntry{\faRunning}{\href{https://youtu.be/dQw4w9WgXcQ}{Urban fitting}.}%
  \end{minipage}
%
  \hspace{+2.00 cm}%
%
  \begin{minipage}[t]{0.42\linewidth}
    \cvsection{\faPeopleCarry}{Associatif}
    \extraEntry{\faCampground}{Scoutisme (en tant que scout et chef).}%
    \extraEntry{\faRobot}{Membre du Lab Robotique à \Epitech}%
    \extraEntry{\faCameraRetro}{Photographe du BDE de l'\EISTI.}%
    \extraEntry{\faVideo}{Responsable \textit{Cinéma \& Séries} à \AirEISTI.}%
  \end{minipage}

  %%%%%%%%%%%%%%%%%%%%%%%%%%%%%%%%%%%%%%%%%%%%%%%%%%

\end{document}


  %%%%%%%%%%%%%%%%%%%%%%%%%%%%%%%%%%%%%%%%%%%%%%%%%%
  %% Title & Personal information.

  \makecvtitle

  %%%%%%%%%%%%%%%%%%%%%%%%%%%%%%%%%%%%%%%%%%%%%%%%%%

  \vspace{-3.50 em}

  %%%%%%%%%%%%%%%%%%%%%%%%%%%%%%%%%%%%%%%%%%%%%%%%%%
  %% Skills section.
  \cvsection{\faCode}{Compétences techniques}

  \skillEntry{\faCoffee}{Écosystème \Java}{\SpringBoot, \RxJava, \Hibernate, \QueryDSL, \Flyway, \Testcontainers, \Gradle.}
  \skillEntry{\faDatabase}{BDD \& Système}{\Postgres, \MongoDB, \ElasticSearch, \Scylla, \Bash, \Docker, \Linux.}
  \skillEntry{\faGlobeAfrica}{Web \& Messaging}{\REST, \OpenAPI, \JWT, \OAuthTwo, \NATS, \PubSub, \React, \TypeScript.}
  \skillEntry{\faProjectDiagram}{Méthodologies}{\HexagonalArch, \CleanCode, \TDD, \Scrum, \Kanban, \DX.}
  \skillEntry{\faTools}{Outils}{\IntelliJ, \Git, \GitHub, \GitHubActions, \Postman, \Lens, \Grafana, \Jira, \LinkLaTeX.}

  %%%%%%%%%%%%%%%%%%%%%%%%%%%%%%%%%%%%%%%%%%%%%%%%%%

  \vspace{\cvSectionSpace}

  %%%%%%%%%%%%%%%%%%%%%%%%%%%%%%%%%%%%%%%%%%%%%%%%%%
  %% Experiences section.
  \cvsection{\faBriefcase}{Expériences professionnelles}

  \experienceEntry{\Happn}{Paris}{\remoteIcon}{Février 2024}{}{Java Software Engineer}{\#Rencontre \#App \#Legacy \faHeart}{%
    \experienceEntryContent{\happnURL}{./img/happn}
    {Développeur dans l'équipe backend d'\Happn, l'application de rencontre qui permet de retrouver les personnes croisées, que ce soit dans la rue, un café, ou les transports.}
    {
        {Développement de nouvelles features phares (\textit{\SmartLOL}, \textit{\PerfectDates}).},
        {Migration des microservices existants vers une \MakeLowercase{\HexagonalArch}.},
        {Dette \& upgrades: \Java{} 11/17$\;$\rightarrow{} 21, \SpringBoot{} 2$\;$\rightarrow{} 3, \Maven{}$\;$\rightarrow{} \Gradle.}
    }
    {\JavaTwentyOne, \SpringBoot, \ElasticSearch, \Scylla, \PubSub, \OpenAPI}
  }

  \experienceEntry{\SetKeeper}{Paris}{\remoteIcon}{Juin 2022}{Novembre 2023}{Java Software Engineer}{\#Cinéma \#Paie \#US \faFilm}{%
    \experienceEntryContent{\setkeeperURL}{./img/setkeeper}
    {Développeur backend dans l'équipe produit de \SetKeeper, une plateforme de gestion, d'organisation et de planification pour les équipes de tournage \href{\setkeeperClientsURL}{(films, séries, TV)}.}
    {
        {Mise en place d'outils et méthodes pour améliorer la \textit{Developer Experience} (\DX).},
        {Intégration de \Google{} (\GoogleDrive{} \& \GoogleContacts) pour du partage de documents générés.},
        {Migration d'une partie du code legacy vers une stack plus à jour.}
    }
    {\JavaSeventeen, \Vertx, \RxJava, \MongoDB, \Maven, \Docker, \OpenAPI}
  }

  \experienceEntry{\Metroscope}{Paris}{\remoteIcon}{Août 2019}{Juin 2022}{Full Stack Developer}{\#Energie \#Maintenance \#\IoT{} \faIndustry}{%
    \experienceEntryContent{\metroscopeURL}{./img/metroscope}
    {Développeur front puis back dans l'équipe produit de \Metroscope, un logiciel d'aide à l'analyse et la détection de défaillances sur les systèmes de production d'énergie.}
    {
        {Réécriture \textit{*from scratch*} de l'application web en \React{} et \TypeScript.},
        {Division du backend monolithique en microservices \& création de nouveaux services.},
        {Passage à une communication évènementielle entre les différents microservices.},
        {Veille et partage de connaissances via l'organisation de \textit{"Backend Chapters"}.}
    }
    {\JavaEleven, \Kotlin, \SpringBoot, \NATS, \Docker, \React, \TypeScript}
  }

  \experienceEntry{\Wemanity}{Paris}{}{Mars 2018}{Juillet 2019}{Full Stack Lead Developer}{\#Banque \#Innovation \#IA \faMoneyBill*}{%
    \experienceEntryContent{\wemanityURL}{./img/wemanity}
    {Lead Developer \RD{} au sein du Lab Innovation \textit{Innov8} de la \SG.}
    {
        {Idéation, création et maintenance d'applications autour des robots \Pepper.},
        {Développement d'un MVP de data-visualisation de graphes complexes en 3D.},
        {Étude et prototypage de services cognitifs (\OCR, \NLU, \STT{} et \Traduction).},
        {Formation et encadrement des developpeurs (stagiaires, alternants et juniors).}
    }
    {\React, \MaterialUI, \Loopback, \Python, \Docker, \GitLab}
  }

  \experienceEntry{\Dhatim}{Paris}{}{Octobre 2016}{Février 2018}{Java Developer}{\#Paie \#Comptabilité \#RH \faFileInvoiceDollar}{%
    \experienceEntryContent{\dhatimURL}{./img/dhatim}
    {Développeur backend dans l'équipe de \textit{\Conciliator}, une solution en SaaS de contrôle de \DSN{} pour les entreprises.}
    {
        {Prototypage et développement des nouvelles fonctionalités, \Scrum{} Master.},
        {Compréhension des besoins pour du support technique auprès des utilisateurs.},
        {Mise en place d'un POC de chatbot \Intercom{} pour du support client simple.}
    }
    {\JavaEight, \Dropwizard, \TestNG, \Postgres, \GitHub, \Jenkins}
  }

  \experienceEntry{\Visian}{Nanterre}{}{Mars 2015}{Octobre 2016}{\IoT{} Lead Developer}{\#\IoT{} \#\RD{} \#Innovation \faMicrochip}{%
    \experienceEntryContent{\visianURL}{./img/visian}
    {Développeur \RD{} puis Lead Developer chez \Visian, la filiale \IoT{} de \NeuronesIT.}
    {
        {Animation d'ateliers d'idéation, compréhension des besoins, veille technique.},
        {Développement de prototypes mêlant \IoT{}, mobile, cloud et data-visualisation.},
        {Encadrement et mentorat des équipes techniques (hardware \& software).}
    }
    {\LanguageCCPlusPlus, \JavaEight, \Android, \Python, \GitHub, \MicrosoftAzure}
  }

  %%%%%%%%%%%%%%%%%%%%%%%%%%%%%%%%%%%%%%%%%%%%%%%%%%

  \vspace{\cvSectionSpace}

  %%%%%%%%%%%%%%%%%%%%%%%%%%%%%%%%%%%%%%%%%%%%%%%%%%
  %% Education section.
  \cvsection{\faGraduationCap}{Parcours scolaire, stages \& alternances}

  \educationEntry{\Epitech}{2011}{2015}{\faUniversity}{Titre d'Expert en Technologie de l'Information}{Toulouse/Paris}{%
    \jobItems{
        {\companyName{\SII}: Prototypage \IoT{} combinant \href{\bitcrazeURL}{drone}, \RaspberryPi{} et divers capteurs (1 an).},
        {\companyName{\Thales}: Étude et déploiement d'\OpenStack{} dans un environnement de tests (6 mois).},
        {\companyName{\Novacom}: Développement d'un ensemble de tests unitaires et fonctionnels (1 an).}
    }
  }
  \educationEntry{\Griffith}{2013}{2014}{\faBeer}{Master I en Computing option Business \& Management}{Dublin}{}
  \educationEntry{\EISTI}{2008}{2011}{\faBrain}{Classes préparatoires Maths Sup/Spé (équivalence L2 Math/Info)}{Pau}{}
  \educationEntry{\Vauvenargues}{}{2008}{\faSchool}{Bac S options Maths \& Sciences de l'Ingénieur}{Aix-en-Provence}{}

  %%%%%%%%%%%%%%%%%%%%%%%%%%%%%%%%%%%%%%%%%%%%%%%%%%

  \vspace{\cvSectionSpace}

  %%%%%%%%%%%%%%%%%%%%%%%%%%%%%%%%%%%%%%%%%%%%%%%%%%
  %% Hobbies & Community sections.
  \begin{minipage}[t]{0.42\linewidth}
    \cvsection{\faTheaterMasks}{Centres d'intérêts}
    \extraEntry{\faPhotoVideo}{Photographie, musique, cinéma.}%
    \extraEntry{\faBook}{Lecture (thrillers, SF, fantasy, comics).}%
    \extraEntry{\faGamepad}{Lego, robotique, gaming, open source.}%
    \extraEntry{\faRunning}{\href{https://youtu.be/dQw4w9WgXcQ}{Urban fitting}.}%
  \end{minipage}
%
  \hspace{+2.00 cm}%
%
  \begin{minipage}[t]{0.42\linewidth}
    \cvsection{\faPeopleCarry}{Associatif}
    \extraEntry{\faCampground}{Scoutisme (en tant que scout et chef).}%
    \extraEntry{\faRobot}{Membre du Lab Robotique à \Epitech}%
    \extraEntry{\faCameraRetro}{Photographe du BDE de l'\EISTI.}%
    \extraEntry{\faVideo}{Responsable \textit{Cinéma \& Séries} à \AirEISTI.}%
  \end{minipage}

  %%%%%%%%%%%%%%%%%%%%%%%%%%%%%%%%%%%%%%%%%%%%%%%%%%

\end{document}


  %%%%%%%%%%%%%%%%%%%%%%%%%%%%%%%%%%%%%%%%%%%%%%%%%%
  %% Title & Personal information.

  \makecvtitle

  %%%%%%%%%%%%%%%%%%%%%%%%%%%%%%%%%%%%%%%%%%%%%%%%%%

  \vspace{-3.50 em}

  %%%%%%%%%%%%%%%%%%%%%%%%%%%%%%%%%%%%%%%%%%%%%%%%%%
  %% Skills section.
  \cvsection{\faCode}{Compétences techniques}

  \skillEntry{\faCoffee}{Écosystème \Java}{\SpringBoot, \RxJava, \Hibernate, \QueryDSL, \Flyway, \Testcontainers, \Gradle.}
  \skillEntry{\faDatabase}{BDD \& Système}{\Postgres, \MongoDB, \ElasticSearch, \Scylla, \Bash, \Docker, \Linux.}
  \skillEntry{\faGlobeAfrica}{Web \& Messaging}{\REST, \OpenAPI, \JWT, \OAuthTwo, \NATS, \PubSub, \React, \TypeScript.}
  \skillEntry{\faProjectDiagram}{Méthodologies}{\HexagonalArch, \CleanCode, \TDD, \Scrum, \Kanban, \DX.}
  \skillEntry{\faTools}{Outils}{\IntelliJ, \Git, \GitHub, \GitHubActions, \Postman, \Lens, \Grafana, \Jira, \LinkLaTeX.}

  %%%%%%%%%%%%%%%%%%%%%%%%%%%%%%%%%%%%%%%%%%%%%%%%%%

  \vspace{\cvSectionSpace}

  %%%%%%%%%%%%%%%%%%%%%%%%%%%%%%%%%%%%%%%%%%%%%%%%%%
  %% Experiences section.
  \cvsection{\faBriefcase}{Expériences professionnelles}

  \experienceEntry{\Happn}{Paris}{\remoteIcon}{Février 2024}{}{Java Software Engineer}{\#Rencontre \#App \#Legacy \faHeart}{%
    \experienceEntryContent{\happnURL}{./img/happn}
    {Développeur dans l'équipe backend d'\Happn, l'application de rencontre qui permet de retrouver les personnes croisées, que ce soit dans la rue, un café, ou les transports.}
    {
        {Développement de nouvelles features phares (\textit{\SmartLOL}, \textit{\PerfectDates}).},
        {Migration des microservices existants vers une \MakeLowercase{\HexagonalArch}.},
        {Dette \& upgrades: \Java{} 11/17$\;$\rightarrow{} 21, \SpringBoot{} 2$\;$\rightarrow{} 3, \Maven{}$\;$\rightarrow{} \Gradle.}
    }
    {\JavaTwentyOne, \SpringBoot, \ElasticSearch, \Scylla, \PubSub, \OpenAPI}
  }

  \experienceEntry{\SetKeeper}{Paris}{\remoteIcon}{Juin 2022}{Novembre 2023}{Java Software Engineer}{\#Cinéma \#Paie \#US \faFilm}{%
    \experienceEntryContent{\setkeeperURL}{./img/setkeeper}
    {Développeur backend dans l'équipe produit de \SetKeeper, une plateforme de gestion, d'organisation et de planification pour les équipes de tournage \href{\setkeeperClientsURL}{(films, séries, TV)}.}
    {
        {Mise en place d'outils et méthodes pour améliorer la \textit{Developer Experience} (\DX).},
        {Intégration de \Google{} (\GoogleDrive{} \& \GoogleContacts) pour du partage de documents générés.},
        {Migration d'une partie du code legacy vers une stack plus à jour.}
    }
    {\JavaSeventeen, \Vertx, \RxJava, \MongoDB, \Maven, \Docker, \OpenAPI}
  }

  \experienceEntry{\Metroscope}{Paris}{\remoteIcon}{Août 2019}{Juin 2022}{Full Stack Developer}{\#Energie \#Maintenance \#\IoT{} \faIndustry}{%
    \experienceEntryContent{\metroscopeURL}{./img/metroscope}
    {Développeur front puis back dans l'équipe produit de \Metroscope, un logiciel d'aide à l'analyse et la détection de défaillances sur les systèmes de production d'énergie.}
    {
        {Réécriture \textit{*from scratch*} de l'application web en \React{} et \TypeScript.},
        {Division du backend monolithique en microservices \& création de nouveaux services.},
        {Passage à une communication évènementielle entre les différents microservices.},
        {Veille et partage de connaissances via l'organisation de \textit{"Backend Chapters"}.}
    }
    {\JavaEleven, \Kotlin, \SpringBoot, \NATS, \Docker, \React, \TypeScript}
  }

  \experienceEntry{\Wemanity}{Paris}{}{Mars 2018}{Juillet 2019}{Full Stack Lead Developer}{\#Banque \#Innovation \#IA \faMoneyBill*}{%
    \experienceEntryContent{\wemanityURL}{./img/wemanity}
    {Lead Developer \RD{} au sein du Lab Innovation \textit{Innov8} de la \SG.}
    {
        {Idéation, création et maintenance d'applications autour des robots \Pepper.},
        {Développement d'un MVP de data-visualisation de graphes complexes en 3D.},
        {Étude et prototypage de services cognitifs (\OCR, \NLU, \STT{} et \Traduction).},
        {Formation et encadrement des developpeurs (stagiaires, alternants et juniors).}
    }
    {\React, \MaterialUI, \Loopback, \Python, \Docker, \GitLab}
  }

  \experienceEntry{\Dhatim}{Paris}{}{Octobre 2016}{Février 2018}{Java Developer}{\#Paie \#Comptabilité \#RH \faFileInvoiceDollar}{%
    \experienceEntryContent{\dhatimURL}{./img/dhatim}
    {Développeur backend dans l'équipe de \textit{\Conciliator}, une solution en SaaS de contrôle de \DSN{} pour les entreprises.}
    {
        {Prototypage et développement des nouvelles fonctionalités, \Scrum{} Master.},
        {Compréhension des besoins pour du support technique auprès des utilisateurs.},
        {Mise en place d'un POC de chatbot \Intercom{} pour du support client simple.}
    }
    {\JavaEight, \Dropwizard, \TestNG, \Postgres, \GitHub, \Jenkins}
  }

  \experienceEntry{\Visian}{Nanterre}{}{Mars 2015}{Octobre 2016}{\IoT{} Lead Developer}{\#\IoT{} \#\RD{} \#Innovation \faMicrochip}{%
    \experienceEntryContent{\visianURL}{./img/visian}
    {Développeur \RD{} puis Lead Developer chez \Visian, la filiale \IoT{} de \NeuronesIT.}
    {
        {Animation d'ateliers d'idéation, compréhension des besoins, veille technique.},
        {Développement de prototypes mêlant \IoT{}, mobile, cloud et data-visualisation.},
        {Encadrement et mentorat des équipes techniques (hardware \& software).}
    }
    {\LanguageCCPlusPlus, \JavaEight, \Android, \Python, \GitHub, \MicrosoftAzure}
  }

  %%%%%%%%%%%%%%%%%%%%%%%%%%%%%%%%%%%%%%%%%%%%%%%%%%

  \vspace{\cvSectionSpace}

  %%%%%%%%%%%%%%%%%%%%%%%%%%%%%%%%%%%%%%%%%%%%%%%%%%
  %% Education section.
  \cvsection{\faGraduationCap}{Parcours scolaire, stages \& alternances}

  \educationEntry{\Epitech}{2011}{2015}{\faUniversity}{Titre d'Expert en Technologie de l'Information}{Toulouse/Paris}{%
    \jobItems{
        {\companyName{\SII}: Prototypage \IoT{} combinant \href{\bitcrazeURL}{drone}, \RaspberryPi{} et divers capteurs (1 an).},
        {\companyName{\Thales}: Étude et déploiement d'\OpenStack{} dans un environnement de tests (6 mois).},
        {\companyName{\Novacom}: Développement d'un ensemble de tests unitaires et fonctionnels (1 an).}
    }
  }
  \educationEntry{\Griffith}{2013}{2014}{\faBeer}{Master I en Computing option Business \& Management}{Dublin}{}
  \educationEntry{\EISTI}{2008}{2011}{\faBrain}{Classes préparatoires Maths Sup/Spé (équivalence L2 Math/Info)}{Pau}{}
  \educationEntry{\Vauvenargues}{}{2008}{\faSchool}{Bac S options Maths \& Sciences de l'Ingénieur}{Aix-en-Provence}{}

  %%%%%%%%%%%%%%%%%%%%%%%%%%%%%%%%%%%%%%%%%%%%%%%%%%

  \vspace{\cvSectionSpace}

  %%%%%%%%%%%%%%%%%%%%%%%%%%%%%%%%%%%%%%%%%%%%%%%%%%
  %% Hobbies & Community sections.
  \begin{minipage}[t]{0.42\linewidth}
    \cvsection{\faTheaterMasks}{Centres d'intérêts}
    \extraEntry{\faPhotoVideo}{Photographie, musique, cinéma.}%
    \extraEntry{\faBook}{Lecture (thrillers, SF, fantasy, comics).}%
    \extraEntry{\faGamepad}{Lego, robotique, gaming, open source.}%
    \extraEntry{\faRunning}{\href{https://youtu.be/dQw4w9WgXcQ}{Urban fitting}.}%
  \end{minipage}
%
  \hspace{+2.00 cm}%
%
  \begin{minipage}[t]{0.42\linewidth}
    \cvsection{\faPeopleCarry}{Associatif}
    \extraEntry{\faCampground}{Scoutisme (en tant que scout et chef).}%
    \extraEntry{\faRobot}{Membre du Lab Robotique à \Epitech}%
    \extraEntry{\faCameraRetro}{Photographe du BDE de l'\EISTI.}%
    \extraEntry{\faVideo}{Responsable \textit{Cinéma \& Séries} à \AirEISTI.}%
  \end{minipage}

  %%%%%%%%%%%%%%%%%%%%%%%%%%%%%%%%%%%%%%%%%%%%%%%%%%

\end{document}


%% Personal information.
\name{\textbf{Bertrand}}{\textbf{Boyer}}
\title{\texorpdfstring{Java Software Engineer\newline\large{\SpringBoot{} - \RD{} - Innovation}}{Java Software Engineer}}
\address{}{}{}
\phone[mobile]{+33 (0) 6 10 04 27 16}
% \phone[fixed]{}
% \phone[fax]{}
\email{boyer.bertrand@gmail.com}
\social[linkedin]{BertrandBoyer}
% \social[twitter]{}
\social[github]{BrtrndB}
\homepage{brtrndb.github.io}
% \photo[64pt][0.5pt]{./img/profile}
% \extrainfo{\href{https://youtu.be/pq31rjX1BMw}{Epitech Promotion 2015}}
\quote{%
  \vspace{-2.50 em}\newline%
  « Basically, I transform bad coffee \faCoffee{} into good code \faKeyboard. »%
}%

\begin{document}

  %%%%%%%%%%%%%%%%%%%%%%%%%%%%%%%%%%%%%%%%%%%%%%%%%%
  %% PDF keywords for machine analysis.
  \hypersetup{
    pdfkeywords = {%
      cv, curriculum vit\ae{}, résumé, innovation,%
      software engineer, senior software engineer,%
      java, java 8, java 17, java 21, spring, springboot,%
      spring boot, spring-boot, gradle, git, docker,%
      clean code, tdd, hexagonal arch, latex%
    }%
  }

  %%%%%%%%%%%%%%%%%%%%%%%%%%%%%%%%%%%%%%%%%%%%%%%%%%
  %% Import custom macros for CV layout.
  % !TEX encoding = UTF-8 Unicode
% !TEX TS-program = LuaLaTeX

%% Packages.
\documentclass[10pt,a4paper]{moderncv}
\usepackage{luatextra}
\usepackage[french]{babel}
\usepackage{geometry}
\usepackage[fixed]{fontawesome5}
\usepackage{wrapfig}
\usepackage{ifthen}
\usepackage{xcolor}
\usepackage{textcomp}
\usepackage{eso-pic}
\usepackage{xargs}
\usepackage{expl3}
\usepackage{multirow}
\usepackage{adjustbox}

% Margin options.
\geometry{top=1.50 cm}
\geometry{bottom=1.50 cm}
\geometry{left=1.50 cm}
\geometry{right=1.50 cm}

%% moderncv configuration options.
\moderncvcolor{blue}
\moderncvstyle{classic}
\moderncvicons{awesome}
\nopagenumbers{}
\renewcommand{\familydefault}{\sfdefault}
\setlength{\hintscolumnwidth}{+3.20 cm} % Column width of cvitem.

%% Page border.
\newlength{\pageborder}
\setlength{\pageborder}{3 pt}
\pagecolor{color1}
\AddToShipoutPictureBG{%
  \AtPageLowerLeft{%
    \color{white}%
    \hspace{\pageborder}%
    \rule[\pageborder]{\paperwidth-2\pageborder}{\paperheight-2\pageborder}%
  }%
}

%% Import aliases.
% !TEX encoding = UTF-8 Unicode
% !TEX TS-program = LuaLaTeX

%% Packages.
\documentclass[10pt,a4paper]{moderncv}
\usepackage{luatextra}
\usepackage[french]{babel}
\usepackage{geometry}
\usepackage[fixed]{fontawesome5}
\usepackage{wrapfig}
\usepackage{ifthen}
\usepackage{xcolor}
\usepackage{textcomp}
\usepackage{eso-pic}
\usepackage{xargs}
\usepackage{expl3}
\usepackage{multirow}
\usepackage{adjustbox}

% Margin options.
\geometry{top=1.50 cm}
\geometry{bottom=1.50 cm}
\geometry{left=1.50 cm}
\geometry{right=1.50 cm}

%% moderncv configuration options.
\moderncvcolor{blue}
\moderncvstyle{classic}
\moderncvicons{awesome}
\nopagenumbers{}
\renewcommand{\familydefault}{\sfdefault}
\setlength{\hintscolumnwidth}{+3.20 cm} % Column width of cvitem.

%% Page border.
\newlength{\pageborder}
\setlength{\pageborder}{3 pt}
\pagecolor{color1}
\AddToShipoutPictureBG{%
  \AtPageLowerLeft{%
    \color{white}%
    \hspace{\pageborder}%
    \rule[\pageborder]{\paperwidth-2\pageborder}{\paperheight-2\pageborder}%
  }%
}

%% Import aliases.
% !TEX encoding = UTF-8 Unicode
% !TEX TS-program = LuaLaTeX

%% Packages.
\documentclass[10pt,a4paper]{moderncv}
\usepackage{luatextra}
\usepackage[french]{babel}
\usepackage{geometry}
\usepackage[fixed]{fontawesome5}
\usepackage{wrapfig}
\usepackage{ifthen}
\usepackage{xcolor}
\usepackage{textcomp}
\usepackage{eso-pic}
\usepackage{xargs}
\usepackage{expl3}
\usepackage{multirow}
\usepackage{adjustbox}

% Margin options.
\geometry{top=1.50 cm}
\geometry{bottom=1.50 cm}
\geometry{left=1.50 cm}
\geometry{right=1.50 cm}

%% moderncv configuration options.
\moderncvcolor{blue}
\moderncvstyle{classic}
\moderncvicons{awesome}
\nopagenumbers{}
\renewcommand{\familydefault}{\sfdefault}
\setlength{\hintscolumnwidth}{+3.20 cm} % Column width of cvitem.

%% Page border.
\newlength{\pageborder}
\setlength{\pageborder}{3 pt}
\pagecolor{color1}
\AddToShipoutPictureBG{%
  \AtPageLowerLeft{%
    \color{white}%
    \hspace{\pageborder}%
    \rule[\pageborder]{\paperwidth-2\pageborder}{\paperheight-2\pageborder}%
  }%
}

%% Import aliases.
\input{./BertrandBoyer.aliases}

%% Personal information.
\name{\textbf{Bertrand}}{\textbf{Boyer}}
\title{\texorpdfstring{Java Software Engineer\newline\large{\SpringBoot{} - \RD{} - Innovation}}{Java Software Engineer}}
\address{}{}{}
\phone[mobile]{+33 (0) 6 10 04 27 16}
% \phone[fixed]{}
% \phone[fax]{}
\email{boyer.bertrand@gmail.com}
\social[linkedin]{BertrandBoyer}
\social[github]{BrtrndB}
\homepage{brtrndb.github.io}
% \social[twitter]{}
% \photo[64pt][0.5pt]{./img/profile}
% \extrainfo{\href{https://youtu.be/pq31rjX1BMw}{Epitech Promotion 2015}}
\quote{%
  \vspace{-2.50 em}\newline%
  « Basically, I transform bad coffee \faCoffee{} into good code \faKeyboard. »%
}%

\begin{document}
  %%%%%%%%%%%%%%%%%%%%%%%%%%%%%%%%%%%%%%%%%%%%%%%%%%
  %% Import custom macros for CV layout.
  \input{./BertrandBoyer.macros}

  %%%%%%%%%%%%%%%%%%%%%%%%%%%%%%%%%%%%%%%%%%%%%%%%%%
  %% Title & Personal information.

  \makecvtitle

  %%%%%%%%%%%%%%%%%%%%%%%%%%%%%%%%%%%%%%%%%%%%%%%%%%

  \vspace{-3.50 em}

  %%%%%%%%%%%%%%%%%%%%%%%%%%%%%%%%%%%%%%%%%%%%%%%%%%
  %% Skills section.
  \cvsection{\faCode}{Compétences techniques}

  \skillEntry{\faCoffee}{Écosystème \Java}{\SpringBoot, \RxJava, \Hibernate, \QueryDSL, \Flyway, \Testcontainers, \Gradle.}
  \skillEntry{\faDatabase}{BDD \& Système}{\Postgres, \MongoDB, \ElasticSearch, \Scylla, \Bash, \Docker, \Linux.}
  \skillEntry{\faGlobeAfrica}{Web \& Messaging}{\REST, \OpenAPI, \JWT, \OAuthTwo, \NATS, \PubSub, \React, \TypeScript.}
  \skillEntry{\faProjectDiagram}{Méthodologies}{\HexagonalArch, \CleanCode, \TDD, \Scrum, \Kanban, \DX.}
  \skillEntry{\faTools}{Outils}{\IntelliJ, \Git, \GitHub, \GitHubActions, \Postman, \Lens, \Grafana, \Jira, \LinkLaTeX.}

  %%%%%%%%%%%%%%%%%%%%%%%%%%%%%%%%%%%%%%%%%%%%%%%%%%

  \vspace{\cvSectionSpace}

  %%%%%%%%%%%%%%%%%%%%%%%%%%%%%%%%%%%%%%%%%%%%%%%%%%
  %% Experiences section.
  \cvsection{\faBriefcase}{Expériences professionnelles}

  \experienceEntry{\Happn}{Paris}{\remoteIcon}{Février 2024}{}{Java Software Engineer}{\#Rencontre \#App \#Legacy \faHeart}{%
    \experienceEntryContent{\happnURL}{./img/happn}
    {Développeur dans l'équipe backend d'\Happn, l'application de rencontre qui permet de retrouver les personnes croisées, que ce soit dans la rue, un café, ou les transports.}
    {
        {Développement de nouvelles features phares (\textit{\SmartLOL}, \textit{\PerfectDates}).},
        {Migration des microservices existants vers une \MakeLowercase{\HexagonalArch}.},
        {Dette \& upgrades: \Java{} 11/17$\;$\rightarrow{} 21, \SpringBoot{} 2$\;$\rightarrow{} 3, \Maven{}$\;$\rightarrow{} \Gradle.}
    }
    {\JavaTwentyOne, \SpringBoot, \ElasticSearch, \Scylla, \PubSub, \OpenAPI}
  }

  \experienceEntry{\SetKeeper}{Paris}{\remoteIcon}{Juin 2022}{Novembre 2023}{Java Software Engineer}{\#Cinéma \#Paie \#US \faFilm}{%
    \experienceEntryContent{\setkeeperURL}{./img/setkeeper}
    {Développeur backend dans l'équipe produit de \SetKeeper, une plateforme de gestion, d'organisation et de planification pour les équipes de tournage \href{\setkeeperClientsURL}{(films, séries, TV)}.}
    {
        {Mise en place d'outils et méthodes pour améliorer la \textit{Developer Experience} (\DX).},
        {Intégration de \Google{} (\GoogleDrive{} \& \GoogleContacts) pour du partage de documents générés.},
        {Migration d'une partie du code legacy vers une stack plus à jour.}
    }
    {\JavaSeventeen, \Vertx, \RxJava, \MongoDB, \Maven, \Docker, \OpenAPI}
  }

  \experienceEntry{\Metroscope}{Paris}{\remoteIcon}{Août 2019}{Juin 2022}{Full Stack Developer}{\#Energie \#Maintenance \#\IoT{} \faIndustry}{%
    \experienceEntryContent{\metroscopeURL}{./img/metroscope}
    {Développeur front puis back dans l'équipe produit de \Metroscope, un logiciel d'aide à l'analyse et la détection de défaillances sur les systèmes de production d'énergie.}
    {
        {Réécriture \textit{*from scratch*} de l'application web en \React{} et \TypeScript.},
        {Division du backend monolithique en microservices \& création de nouveaux services.},
        {Passage à une communication évènementielle entre les différents microservices.},
        {Veille et partage de connaissances via l'organisation de \textit{"Backend Chapters"}.}
    }
    {\JavaEleven, \Kotlin, \SpringBoot, \NATS, \Docker, \React, \TypeScript}
  }

  \experienceEntry{\Wemanity}{Paris}{}{Mars 2018}{Juillet 2019}{Full Stack Lead Developer}{\#Banque \#Innovation \#IA \faMoneyBill*}{%
    \experienceEntryContent{\wemanityURL}{./img/wemanity}
    {Lead Developer \RD{} au sein du Lab Innovation \textit{Innov8} de la \SG.}
    {
        {Idéation, création et maintenance d'applications autour des robots \Pepper.},
        {Développement d'un MVP de data-visualisation de graphes complexes en 3D.},
        {Étude et prototypage de services cognitifs (\OCR, \NLU, \STT{} et \Traduction).},
        {Formation et encadrement des developpeurs (stagiaires, alternants et juniors).}
    }
    {\React, \MaterialUI, \Loopback, \Python, \Docker, \GitLab}
  }

  \experienceEntry{\Dhatim}{Paris}{}{Octobre 2016}{Février 2018}{Java Developer}{\#Paie \#Comptabilité \#RH \faFileInvoiceDollar}{%
    \experienceEntryContent{\dhatimURL}{./img/dhatim}
    {Développeur backend dans l'équipe de \textit{\Conciliator}, une solution en SaaS de contrôle de \DSN{} pour les entreprises.}
    {
        {Prototypage et développement des nouvelles fonctionalités, \Scrum{} Master.},
        {Compréhension des besoins pour du support technique auprès des utilisateurs.},
        {Mise en place d'un POC de chatbot \Intercom{} pour du support client simple.}
    }
    {\JavaEight, \Dropwizard, \TestNG, \Postgres, \GitHub, \Jenkins}
  }

  \experienceEntry{\Visian}{Nanterre}{}{Mars 2015}{Octobre 2016}{\IoT{} Lead Developer}{\#\IoT{} \#\RD{} \#Innovation \faMicrochip}{%
    \experienceEntryContent{\visianURL}{./img/visian}
    {Développeur \RD{} puis Lead Developer chez \Visian, la filiale \IoT{} de \NeuronesIT.}
    {
        {Animation d'ateliers d'idéation, compréhension des besoins, veille technique.},
        {Développement de prototypes mêlant \IoT{}, mobile, cloud et data-visualisation.},
        {Encadrement et mentorat des équipes techniques (hardware \& software).}
    }
    {\LanguageCCPlusPlus, \JavaEight, \Android, \Python, \GitHub, \MicrosoftAzure}
  }

  %%%%%%%%%%%%%%%%%%%%%%%%%%%%%%%%%%%%%%%%%%%%%%%%%%

  \vspace{\cvSectionSpace}

  %%%%%%%%%%%%%%%%%%%%%%%%%%%%%%%%%%%%%%%%%%%%%%%%%%
  %% Education section.
  \cvsection{\faGraduationCap}{Parcours scolaire, stages \& alternances}

  \educationEntry{\Epitech}{2011}{2015}{\faUniversity}{Titre d'Expert en Technologie de l'Information}{Toulouse/Paris}{%
    \jobItems{
        {\companyName{\SII}: Prototypage \IoT{} combinant \href{\bitcrazeURL}{drone}, \RaspberryPi{} et divers capteurs (1 an).},
        {\companyName{\Thales}: Étude et déploiement d'\OpenStack{} dans un environnement de tests (6 mois).},
        {\companyName{\Novacom}: Développement d'un ensemble de tests unitaires et fonctionnels (1 an).}
    }
  }
  \educationEntry{\Griffith}{2013}{2014}{\faBeer}{Master I en Computing option Business \& Management}{Dublin}{}
  \educationEntry{\EISTI}{2008}{2011}{\faBrain}{Classes préparatoires Maths Sup/Spé (équivalence L2 Math/Info)}{Pau}{}
  \educationEntry{\Vauvenargues}{}{2008}{\faSchool}{Bac S options Maths \& Sciences de l'Ingénieur}{Aix-en-Provence}{}

  %%%%%%%%%%%%%%%%%%%%%%%%%%%%%%%%%%%%%%%%%%%%%%%%%%

  \vspace{\cvSectionSpace}

  %%%%%%%%%%%%%%%%%%%%%%%%%%%%%%%%%%%%%%%%%%%%%%%%%%
  %% Hobbies & Community sections.
  \begin{minipage}[t]{0.42\linewidth}
    \cvsection{\faTheaterMasks}{Centres d'intérêts}
    \extraEntry{\faPhotoVideo}{Photographie, musique, cinéma.}%
    \extraEntry{\faBook}{Lecture (thrillers, SF, fantasy, comics).}%
    \extraEntry{\faGamepad}{Lego, robotique, gaming, open source.}%
    \extraEntry{\faRunning}{\href{https://youtu.be/dQw4w9WgXcQ}{Urban fitting}.}%
  \end{minipage}
%
  \hspace{+2.00 cm}%
%
  \begin{minipage}[t]{0.42\linewidth}
    \cvsection{\faPeopleCarry}{Associatif}
    \extraEntry{\faCampground}{Scoutisme (en tant que scout et chef).}%
    \extraEntry{\faRobot}{Membre du Lab Robotique à \Epitech}%
    \extraEntry{\faCameraRetro}{Photographe du BDE de l'\EISTI.}%
    \extraEntry{\faVideo}{Responsable \textit{Cinéma \& Séries} à \AirEISTI.}%
  \end{minipage}

  %%%%%%%%%%%%%%%%%%%%%%%%%%%%%%%%%%%%%%%%%%%%%%%%%%

\end{document}


%% Personal information.
\name{\textbf{Bertrand}}{\textbf{Boyer}}
\title{\texorpdfstring{Java Software Engineer\newline\large{\SpringBoot{} - \RD{} - Innovation}}{Java Software Engineer}}
\address{}{}{}
\phone[mobile]{+33 (0) 6 10 04 27 16}
% \phone[fixed]{}
% \phone[fax]{}
\email{boyer.bertrand@gmail.com}
\social[linkedin]{BertrandBoyer}
\social[github]{BrtrndB}
\homepage{brtrndb.github.io}
% \social[twitter]{}
% \photo[64pt][0.5pt]{./img/profile}
% \extrainfo{\href{https://youtu.be/pq31rjX1BMw}{Epitech Promotion 2015}}
\quote{%
  \vspace{-2.50 em}\newline%
  « Basically, I transform bad coffee \faCoffee{} into good code \faKeyboard. »%
}%

\begin{document}
  %%%%%%%%%%%%%%%%%%%%%%%%%%%%%%%%%%%%%%%%%%%%%%%%%%
  %% Import custom macros for CV layout.
  % !TEX encoding = UTF-8 Unicode
% !TEX TS-program = LuaLaTeX

%% Packages.
\documentclass[10pt,a4paper]{moderncv}
\usepackage{luatextra}
\usepackage[french]{babel}
\usepackage{geometry}
\usepackage[fixed]{fontawesome5}
\usepackage{wrapfig}
\usepackage{ifthen}
\usepackage{xcolor}
\usepackage{textcomp}
\usepackage{eso-pic}
\usepackage{xargs}
\usepackage{expl3}
\usepackage{multirow}
\usepackage{adjustbox}

% Margin options.
\geometry{top=1.50 cm}
\geometry{bottom=1.50 cm}
\geometry{left=1.50 cm}
\geometry{right=1.50 cm}

%% moderncv configuration options.
\moderncvcolor{blue}
\moderncvstyle{classic}
\moderncvicons{awesome}
\nopagenumbers{}
\renewcommand{\familydefault}{\sfdefault}
\setlength{\hintscolumnwidth}{+3.20 cm} % Column width of cvitem.

%% Page border.
\newlength{\pageborder}
\setlength{\pageborder}{3 pt}
\pagecolor{color1}
\AddToShipoutPictureBG{%
  \AtPageLowerLeft{%
    \color{white}%
    \hspace{\pageborder}%
    \rule[\pageborder]{\paperwidth-2\pageborder}{\paperheight-2\pageborder}%
  }%
}

%% Import aliases.
\input{./BertrandBoyer.aliases}

%% Personal information.
\name{\textbf{Bertrand}}{\textbf{Boyer}}
\title{\texorpdfstring{Java Software Engineer\newline\large{\SpringBoot{} - \RD{} - Innovation}}{Java Software Engineer}}
\address{}{}{}
\phone[mobile]{+33 (0) 6 10 04 27 16}
% \phone[fixed]{}
% \phone[fax]{}
\email{boyer.bertrand@gmail.com}
\social[linkedin]{BertrandBoyer}
\social[github]{BrtrndB}
\homepage{brtrndb.github.io}
% \social[twitter]{}
% \photo[64pt][0.5pt]{./img/profile}
% \extrainfo{\href{https://youtu.be/pq31rjX1BMw}{Epitech Promotion 2015}}
\quote{%
  \vspace{-2.50 em}\newline%
  « Basically, I transform bad coffee \faCoffee{} into good code \faKeyboard. »%
}%

\begin{document}
  %%%%%%%%%%%%%%%%%%%%%%%%%%%%%%%%%%%%%%%%%%%%%%%%%%
  %% Import custom macros for CV layout.
  \input{./BertrandBoyer.macros}

  %%%%%%%%%%%%%%%%%%%%%%%%%%%%%%%%%%%%%%%%%%%%%%%%%%
  %% Title & Personal information.

  \makecvtitle

  %%%%%%%%%%%%%%%%%%%%%%%%%%%%%%%%%%%%%%%%%%%%%%%%%%

  \vspace{-3.50 em}

  %%%%%%%%%%%%%%%%%%%%%%%%%%%%%%%%%%%%%%%%%%%%%%%%%%
  %% Skills section.
  \cvsection{\faCode}{Compétences techniques}

  \skillEntry{\faCoffee}{Écosystème \Java}{\SpringBoot, \RxJava, \Hibernate, \QueryDSL, \Flyway, \Testcontainers, \Gradle.}
  \skillEntry{\faDatabase}{BDD \& Système}{\Postgres, \MongoDB, \ElasticSearch, \Scylla, \Bash, \Docker, \Linux.}
  \skillEntry{\faGlobeAfrica}{Web \& Messaging}{\REST, \OpenAPI, \JWT, \OAuthTwo, \NATS, \PubSub, \React, \TypeScript.}
  \skillEntry{\faProjectDiagram}{Méthodologies}{\HexagonalArch, \CleanCode, \TDD, \Scrum, \Kanban, \DX.}
  \skillEntry{\faTools}{Outils}{\IntelliJ, \Git, \GitHub, \GitHubActions, \Postman, \Lens, \Grafana, \Jira, \LinkLaTeX.}

  %%%%%%%%%%%%%%%%%%%%%%%%%%%%%%%%%%%%%%%%%%%%%%%%%%

  \vspace{\cvSectionSpace}

  %%%%%%%%%%%%%%%%%%%%%%%%%%%%%%%%%%%%%%%%%%%%%%%%%%
  %% Experiences section.
  \cvsection{\faBriefcase}{Expériences professionnelles}

  \experienceEntry{\Happn}{Paris}{\remoteIcon}{Février 2024}{}{Java Software Engineer}{\#Rencontre \#App \#Legacy \faHeart}{%
    \experienceEntryContent{\happnURL}{./img/happn}
    {Développeur dans l'équipe backend d'\Happn, l'application de rencontre qui permet de retrouver les personnes croisées, que ce soit dans la rue, un café, ou les transports.}
    {
        {Développement de nouvelles features phares (\textit{\SmartLOL}, \textit{\PerfectDates}).},
        {Migration des microservices existants vers une \MakeLowercase{\HexagonalArch}.},
        {Dette \& upgrades: \Java{} 11/17$\;$\rightarrow{} 21, \SpringBoot{} 2$\;$\rightarrow{} 3, \Maven{}$\;$\rightarrow{} \Gradle.}
    }
    {\JavaTwentyOne, \SpringBoot, \ElasticSearch, \Scylla, \PubSub, \OpenAPI}
  }

  \experienceEntry{\SetKeeper}{Paris}{\remoteIcon}{Juin 2022}{Novembre 2023}{Java Software Engineer}{\#Cinéma \#Paie \#US \faFilm}{%
    \experienceEntryContent{\setkeeperURL}{./img/setkeeper}
    {Développeur backend dans l'équipe produit de \SetKeeper, une plateforme de gestion, d'organisation et de planification pour les équipes de tournage \href{\setkeeperClientsURL}{(films, séries, TV)}.}
    {
        {Mise en place d'outils et méthodes pour améliorer la \textit{Developer Experience} (\DX).},
        {Intégration de \Google{} (\GoogleDrive{} \& \GoogleContacts) pour du partage de documents générés.},
        {Migration d'une partie du code legacy vers une stack plus à jour.}
    }
    {\JavaSeventeen, \Vertx, \RxJava, \MongoDB, \Maven, \Docker, \OpenAPI}
  }

  \experienceEntry{\Metroscope}{Paris}{\remoteIcon}{Août 2019}{Juin 2022}{Full Stack Developer}{\#Energie \#Maintenance \#\IoT{} \faIndustry}{%
    \experienceEntryContent{\metroscopeURL}{./img/metroscope}
    {Développeur front puis back dans l'équipe produit de \Metroscope, un logiciel d'aide à l'analyse et la détection de défaillances sur les systèmes de production d'énergie.}
    {
        {Réécriture \textit{*from scratch*} de l'application web en \React{} et \TypeScript.},
        {Division du backend monolithique en microservices \& création de nouveaux services.},
        {Passage à une communication évènementielle entre les différents microservices.},
        {Veille et partage de connaissances via l'organisation de \textit{"Backend Chapters"}.}
    }
    {\JavaEleven, \Kotlin, \SpringBoot, \NATS, \Docker, \React, \TypeScript}
  }

  \experienceEntry{\Wemanity}{Paris}{}{Mars 2018}{Juillet 2019}{Full Stack Lead Developer}{\#Banque \#Innovation \#IA \faMoneyBill*}{%
    \experienceEntryContent{\wemanityURL}{./img/wemanity}
    {Lead Developer \RD{} au sein du Lab Innovation \textit{Innov8} de la \SG.}
    {
        {Idéation, création et maintenance d'applications autour des robots \Pepper.},
        {Développement d'un MVP de data-visualisation de graphes complexes en 3D.},
        {Étude et prototypage de services cognitifs (\OCR, \NLU, \STT{} et \Traduction).},
        {Formation et encadrement des developpeurs (stagiaires, alternants et juniors).}
    }
    {\React, \MaterialUI, \Loopback, \Python, \Docker, \GitLab}
  }

  \experienceEntry{\Dhatim}{Paris}{}{Octobre 2016}{Février 2018}{Java Developer}{\#Paie \#Comptabilité \#RH \faFileInvoiceDollar}{%
    \experienceEntryContent{\dhatimURL}{./img/dhatim}
    {Développeur backend dans l'équipe de \textit{\Conciliator}, une solution en SaaS de contrôle de \DSN{} pour les entreprises.}
    {
        {Prototypage et développement des nouvelles fonctionalités, \Scrum{} Master.},
        {Compréhension des besoins pour du support technique auprès des utilisateurs.},
        {Mise en place d'un POC de chatbot \Intercom{} pour du support client simple.}
    }
    {\JavaEight, \Dropwizard, \TestNG, \Postgres, \GitHub, \Jenkins}
  }

  \experienceEntry{\Visian}{Nanterre}{}{Mars 2015}{Octobre 2016}{\IoT{} Lead Developer}{\#\IoT{} \#\RD{} \#Innovation \faMicrochip}{%
    \experienceEntryContent{\visianURL}{./img/visian}
    {Développeur \RD{} puis Lead Developer chez \Visian, la filiale \IoT{} de \NeuronesIT.}
    {
        {Animation d'ateliers d'idéation, compréhension des besoins, veille technique.},
        {Développement de prototypes mêlant \IoT{}, mobile, cloud et data-visualisation.},
        {Encadrement et mentorat des équipes techniques (hardware \& software).}
    }
    {\LanguageCCPlusPlus, \JavaEight, \Android, \Python, \GitHub, \MicrosoftAzure}
  }

  %%%%%%%%%%%%%%%%%%%%%%%%%%%%%%%%%%%%%%%%%%%%%%%%%%

  \vspace{\cvSectionSpace}

  %%%%%%%%%%%%%%%%%%%%%%%%%%%%%%%%%%%%%%%%%%%%%%%%%%
  %% Education section.
  \cvsection{\faGraduationCap}{Parcours scolaire, stages \& alternances}

  \educationEntry{\Epitech}{2011}{2015}{\faUniversity}{Titre d'Expert en Technologie de l'Information}{Toulouse/Paris}{%
    \jobItems{
        {\companyName{\SII}: Prototypage \IoT{} combinant \href{\bitcrazeURL}{drone}, \RaspberryPi{} et divers capteurs (1 an).},
        {\companyName{\Thales}: Étude et déploiement d'\OpenStack{} dans un environnement de tests (6 mois).},
        {\companyName{\Novacom}: Développement d'un ensemble de tests unitaires et fonctionnels (1 an).}
    }
  }
  \educationEntry{\Griffith}{2013}{2014}{\faBeer}{Master I en Computing option Business \& Management}{Dublin}{}
  \educationEntry{\EISTI}{2008}{2011}{\faBrain}{Classes préparatoires Maths Sup/Spé (équivalence L2 Math/Info)}{Pau}{}
  \educationEntry{\Vauvenargues}{}{2008}{\faSchool}{Bac S options Maths \& Sciences de l'Ingénieur}{Aix-en-Provence}{}

  %%%%%%%%%%%%%%%%%%%%%%%%%%%%%%%%%%%%%%%%%%%%%%%%%%

  \vspace{\cvSectionSpace}

  %%%%%%%%%%%%%%%%%%%%%%%%%%%%%%%%%%%%%%%%%%%%%%%%%%
  %% Hobbies & Community sections.
  \begin{minipage}[t]{0.42\linewidth}
    \cvsection{\faTheaterMasks}{Centres d'intérêts}
    \extraEntry{\faPhotoVideo}{Photographie, musique, cinéma.}%
    \extraEntry{\faBook}{Lecture (thrillers, SF, fantasy, comics).}%
    \extraEntry{\faGamepad}{Lego, robotique, gaming, open source.}%
    \extraEntry{\faRunning}{\href{https://youtu.be/dQw4w9WgXcQ}{Urban fitting}.}%
  \end{minipage}
%
  \hspace{+2.00 cm}%
%
  \begin{minipage}[t]{0.42\linewidth}
    \cvsection{\faPeopleCarry}{Associatif}
    \extraEntry{\faCampground}{Scoutisme (en tant que scout et chef).}%
    \extraEntry{\faRobot}{Membre du Lab Robotique à \Epitech}%
    \extraEntry{\faCameraRetro}{Photographe du BDE de l'\EISTI.}%
    \extraEntry{\faVideo}{Responsable \textit{Cinéma \& Séries} à \AirEISTI.}%
  \end{minipage}

  %%%%%%%%%%%%%%%%%%%%%%%%%%%%%%%%%%%%%%%%%%%%%%%%%%

\end{document}


  %%%%%%%%%%%%%%%%%%%%%%%%%%%%%%%%%%%%%%%%%%%%%%%%%%
  %% Title & Personal information.

  \makecvtitle

  %%%%%%%%%%%%%%%%%%%%%%%%%%%%%%%%%%%%%%%%%%%%%%%%%%

  \vspace{-3.50 em}

  %%%%%%%%%%%%%%%%%%%%%%%%%%%%%%%%%%%%%%%%%%%%%%%%%%
  %% Skills section.
  \cvsection{\faCode}{Compétences techniques}

  \skillEntry{\faCoffee}{Écosystème \Java}{\SpringBoot, \RxJava, \Hibernate, \QueryDSL, \Flyway, \Testcontainers, \Gradle.}
  \skillEntry{\faDatabase}{BDD \& Système}{\Postgres, \MongoDB, \ElasticSearch, \Scylla, \Bash, \Docker, \Linux.}
  \skillEntry{\faGlobeAfrica}{Web \& Messaging}{\REST, \OpenAPI, \JWT, \OAuthTwo, \NATS, \PubSub, \React, \TypeScript.}
  \skillEntry{\faProjectDiagram}{Méthodologies}{\HexagonalArch, \CleanCode, \TDD, \Scrum, \Kanban, \DX.}
  \skillEntry{\faTools}{Outils}{\IntelliJ, \Git, \GitHub, \GitHubActions, \Postman, \Lens, \Grafana, \Jira, \LinkLaTeX.}

  %%%%%%%%%%%%%%%%%%%%%%%%%%%%%%%%%%%%%%%%%%%%%%%%%%

  \vspace{\cvSectionSpace}

  %%%%%%%%%%%%%%%%%%%%%%%%%%%%%%%%%%%%%%%%%%%%%%%%%%
  %% Experiences section.
  \cvsection{\faBriefcase}{Expériences professionnelles}

  \experienceEntry{\Happn}{Paris}{\remoteIcon}{Février 2024}{}{Java Software Engineer}{\#Rencontre \#App \#Legacy \faHeart}{%
    \experienceEntryContent{\happnURL}{./img/happn}
    {Développeur dans l'équipe backend d'\Happn, l'application de rencontre qui permet de retrouver les personnes croisées, que ce soit dans la rue, un café, ou les transports.}
    {
        {Développement de nouvelles features phares (\textit{\SmartLOL}, \textit{\PerfectDates}).},
        {Migration des microservices existants vers une \MakeLowercase{\HexagonalArch}.},
        {Dette \& upgrades: \Java{} 11/17$\;$\rightarrow{} 21, \SpringBoot{} 2$\;$\rightarrow{} 3, \Maven{}$\;$\rightarrow{} \Gradle.}
    }
    {\JavaTwentyOne, \SpringBoot, \ElasticSearch, \Scylla, \PubSub, \OpenAPI}
  }

  \experienceEntry{\SetKeeper}{Paris}{\remoteIcon}{Juin 2022}{Novembre 2023}{Java Software Engineer}{\#Cinéma \#Paie \#US \faFilm}{%
    \experienceEntryContent{\setkeeperURL}{./img/setkeeper}
    {Développeur backend dans l'équipe produit de \SetKeeper, une plateforme de gestion, d'organisation et de planification pour les équipes de tournage \href{\setkeeperClientsURL}{(films, séries, TV)}.}
    {
        {Mise en place d'outils et méthodes pour améliorer la \textit{Developer Experience} (\DX).},
        {Intégration de \Google{} (\GoogleDrive{} \& \GoogleContacts) pour du partage de documents générés.},
        {Migration d'une partie du code legacy vers une stack plus à jour.}
    }
    {\JavaSeventeen, \Vertx, \RxJava, \MongoDB, \Maven, \Docker, \OpenAPI}
  }

  \experienceEntry{\Metroscope}{Paris}{\remoteIcon}{Août 2019}{Juin 2022}{Full Stack Developer}{\#Energie \#Maintenance \#\IoT{} \faIndustry}{%
    \experienceEntryContent{\metroscopeURL}{./img/metroscope}
    {Développeur front puis back dans l'équipe produit de \Metroscope, un logiciel d'aide à l'analyse et la détection de défaillances sur les systèmes de production d'énergie.}
    {
        {Réécriture \textit{*from scratch*} de l'application web en \React{} et \TypeScript.},
        {Division du backend monolithique en microservices \& création de nouveaux services.},
        {Passage à une communication évènementielle entre les différents microservices.},
        {Veille et partage de connaissances via l'organisation de \textit{"Backend Chapters"}.}
    }
    {\JavaEleven, \Kotlin, \SpringBoot, \NATS, \Docker, \React, \TypeScript}
  }

  \experienceEntry{\Wemanity}{Paris}{}{Mars 2018}{Juillet 2019}{Full Stack Lead Developer}{\#Banque \#Innovation \#IA \faMoneyBill*}{%
    \experienceEntryContent{\wemanityURL}{./img/wemanity}
    {Lead Developer \RD{} au sein du Lab Innovation \textit{Innov8} de la \SG.}
    {
        {Idéation, création et maintenance d'applications autour des robots \Pepper.},
        {Développement d'un MVP de data-visualisation de graphes complexes en 3D.},
        {Étude et prototypage de services cognitifs (\OCR, \NLU, \STT{} et \Traduction).},
        {Formation et encadrement des developpeurs (stagiaires, alternants et juniors).}
    }
    {\React, \MaterialUI, \Loopback, \Python, \Docker, \GitLab}
  }

  \experienceEntry{\Dhatim}{Paris}{}{Octobre 2016}{Février 2018}{Java Developer}{\#Paie \#Comptabilité \#RH \faFileInvoiceDollar}{%
    \experienceEntryContent{\dhatimURL}{./img/dhatim}
    {Développeur backend dans l'équipe de \textit{\Conciliator}, une solution en SaaS de contrôle de \DSN{} pour les entreprises.}
    {
        {Prototypage et développement des nouvelles fonctionalités, \Scrum{} Master.},
        {Compréhension des besoins pour du support technique auprès des utilisateurs.},
        {Mise en place d'un POC de chatbot \Intercom{} pour du support client simple.}
    }
    {\JavaEight, \Dropwizard, \TestNG, \Postgres, \GitHub, \Jenkins}
  }

  \experienceEntry{\Visian}{Nanterre}{}{Mars 2015}{Octobre 2016}{\IoT{} Lead Developer}{\#\IoT{} \#\RD{} \#Innovation \faMicrochip}{%
    \experienceEntryContent{\visianURL}{./img/visian}
    {Développeur \RD{} puis Lead Developer chez \Visian, la filiale \IoT{} de \NeuronesIT.}
    {
        {Animation d'ateliers d'idéation, compréhension des besoins, veille technique.},
        {Développement de prototypes mêlant \IoT{}, mobile, cloud et data-visualisation.},
        {Encadrement et mentorat des équipes techniques (hardware \& software).}
    }
    {\LanguageCCPlusPlus, \JavaEight, \Android, \Python, \GitHub, \MicrosoftAzure}
  }

  %%%%%%%%%%%%%%%%%%%%%%%%%%%%%%%%%%%%%%%%%%%%%%%%%%

  \vspace{\cvSectionSpace}

  %%%%%%%%%%%%%%%%%%%%%%%%%%%%%%%%%%%%%%%%%%%%%%%%%%
  %% Education section.
  \cvsection{\faGraduationCap}{Parcours scolaire, stages \& alternances}

  \educationEntry{\Epitech}{2011}{2015}{\faUniversity}{Titre d'Expert en Technologie de l'Information}{Toulouse/Paris}{%
    \jobItems{
        {\companyName{\SII}: Prototypage \IoT{} combinant \href{\bitcrazeURL}{drone}, \RaspberryPi{} et divers capteurs (1 an).},
        {\companyName{\Thales}: Étude et déploiement d'\OpenStack{} dans un environnement de tests (6 mois).},
        {\companyName{\Novacom}: Développement d'un ensemble de tests unitaires et fonctionnels (1 an).}
    }
  }
  \educationEntry{\Griffith}{2013}{2014}{\faBeer}{Master I en Computing option Business \& Management}{Dublin}{}
  \educationEntry{\EISTI}{2008}{2011}{\faBrain}{Classes préparatoires Maths Sup/Spé (équivalence L2 Math/Info)}{Pau}{}
  \educationEntry{\Vauvenargues}{}{2008}{\faSchool}{Bac S options Maths \& Sciences de l'Ingénieur}{Aix-en-Provence}{}

  %%%%%%%%%%%%%%%%%%%%%%%%%%%%%%%%%%%%%%%%%%%%%%%%%%

  \vspace{\cvSectionSpace}

  %%%%%%%%%%%%%%%%%%%%%%%%%%%%%%%%%%%%%%%%%%%%%%%%%%
  %% Hobbies & Community sections.
  \begin{minipage}[t]{0.42\linewidth}
    \cvsection{\faTheaterMasks}{Centres d'intérêts}
    \extraEntry{\faPhotoVideo}{Photographie, musique, cinéma.}%
    \extraEntry{\faBook}{Lecture (thrillers, SF, fantasy, comics).}%
    \extraEntry{\faGamepad}{Lego, robotique, gaming, open source.}%
    \extraEntry{\faRunning}{\href{https://youtu.be/dQw4w9WgXcQ}{Urban fitting}.}%
  \end{minipage}
%
  \hspace{+2.00 cm}%
%
  \begin{minipage}[t]{0.42\linewidth}
    \cvsection{\faPeopleCarry}{Associatif}
    \extraEntry{\faCampground}{Scoutisme (en tant que scout et chef).}%
    \extraEntry{\faRobot}{Membre du Lab Robotique à \Epitech}%
    \extraEntry{\faCameraRetro}{Photographe du BDE de l'\EISTI.}%
    \extraEntry{\faVideo}{Responsable \textit{Cinéma \& Séries} à \AirEISTI.}%
  \end{minipage}

  %%%%%%%%%%%%%%%%%%%%%%%%%%%%%%%%%%%%%%%%%%%%%%%%%%

\end{document}


%% Personal information.
\name{\textbf{Bertrand}}{\textbf{Boyer}}
\title{\texorpdfstring{Java Software Engineer\newline\large{\SpringBoot{} - \RD{} - Innovation}}{Java Software Engineer}}
\address{}{}{}
\phone[mobile]{+33 (0) 6 10 04 27 16}
% \phone[fixed]{}
% \phone[fax]{}
\email{boyer.bertrand@gmail.com}
\social[linkedin]{BertrandBoyer}
\social[github]{BrtrndB}
\homepage{brtrndb.github.io}
% \social[twitter]{}
% \photo[64pt][0.5pt]{./img/profile}
% \extrainfo{\href{https://youtu.be/pq31rjX1BMw}{Epitech Promotion 2015}}
\quote{%
  \vspace{-2.50 em}\newline%
  « Basically, I transform bad coffee \faCoffee{} into good code \faKeyboard. »%
}%

\begin{document}
  %%%%%%%%%%%%%%%%%%%%%%%%%%%%%%%%%%%%%%%%%%%%%%%%%%
  %% Import custom macros for CV layout.
  % !TEX encoding = UTF-8 Unicode
% !TEX TS-program = LuaLaTeX

%% Packages.
\documentclass[10pt,a4paper]{moderncv}
\usepackage{luatextra}
\usepackage[french]{babel}
\usepackage{geometry}
\usepackage[fixed]{fontawesome5}
\usepackage{wrapfig}
\usepackage{ifthen}
\usepackage{xcolor}
\usepackage{textcomp}
\usepackage{eso-pic}
\usepackage{xargs}
\usepackage{expl3}
\usepackage{multirow}
\usepackage{adjustbox}

% Margin options.
\geometry{top=1.50 cm}
\geometry{bottom=1.50 cm}
\geometry{left=1.50 cm}
\geometry{right=1.50 cm}

%% moderncv configuration options.
\moderncvcolor{blue}
\moderncvstyle{classic}
\moderncvicons{awesome}
\nopagenumbers{}
\renewcommand{\familydefault}{\sfdefault}
\setlength{\hintscolumnwidth}{+3.20 cm} % Column width of cvitem.

%% Page border.
\newlength{\pageborder}
\setlength{\pageborder}{3 pt}
\pagecolor{color1}
\AddToShipoutPictureBG{%
  \AtPageLowerLeft{%
    \color{white}%
    \hspace{\pageborder}%
    \rule[\pageborder]{\paperwidth-2\pageborder}{\paperheight-2\pageborder}%
  }%
}

%% Import aliases.
% !TEX encoding = UTF-8 Unicode
% !TEX TS-program = LuaLaTeX

%% Packages.
\documentclass[10pt,a4paper]{moderncv}
\usepackage{luatextra}
\usepackage[french]{babel}
\usepackage{geometry}
\usepackage[fixed]{fontawesome5}
\usepackage{wrapfig}
\usepackage{ifthen}
\usepackage{xcolor}
\usepackage{textcomp}
\usepackage{eso-pic}
\usepackage{xargs}
\usepackage{expl3}
\usepackage{multirow}
\usepackage{adjustbox}

% Margin options.
\geometry{top=1.50 cm}
\geometry{bottom=1.50 cm}
\geometry{left=1.50 cm}
\geometry{right=1.50 cm}

%% moderncv configuration options.
\moderncvcolor{blue}
\moderncvstyle{classic}
\moderncvicons{awesome}
\nopagenumbers{}
\renewcommand{\familydefault}{\sfdefault}
\setlength{\hintscolumnwidth}{+3.20 cm} % Column width of cvitem.

%% Page border.
\newlength{\pageborder}
\setlength{\pageborder}{3 pt}
\pagecolor{color1}
\AddToShipoutPictureBG{%
  \AtPageLowerLeft{%
    \color{white}%
    \hspace{\pageborder}%
    \rule[\pageborder]{\paperwidth-2\pageborder}{\paperheight-2\pageborder}%
  }%
}

%% Import aliases.
\input{./BertrandBoyer.aliases}

%% Personal information.
\name{\textbf{Bertrand}}{\textbf{Boyer}}
\title{\texorpdfstring{Java Software Engineer\newline\large{\SpringBoot{} - \RD{} - Innovation}}{Java Software Engineer}}
\address{}{}{}
\phone[mobile]{+33 (0) 6 10 04 27 16}
% \phone[fixed]{}
% \phone[fax]{}
\email{boyer.bertrand@gmail.com}
\social[linkedin]{BertrandBoyer}
\social[github]{BrtrndB}
\homepage{brtrndb.github.io}
% \social[twitter]{}
% \photo[64pt][0.5pt]{./img/profile}
% \extrainfo{\href{https://youtu.be/pq31rjX1BMw}{Epitech Promotion 2015}}
\quote{%
  \vspace{-2.50 em}\newline%
  « Basically, I transform bad coffee \faCoffee{} into good code \faKeyboard. »%
}%

\begin{document}
  %%%%%%%%%%%%%%%%%%%%%%%%%%%%%%%%%%%%%%%%%%%%%%%%%%
  %% Import custom macros for CV layout.
  \input{./BertrandBoyer.macros}

  %%%%%%%%%%%%%%%%%%%%%%%%%%%%%%%%%%%%%%%%%%%%%%%%%%
  %% Title & Personal information.

  \makecvtitle

  %%%%%%%%%%%%%%%%%%%%%%%%%%%%%%%%%%%%%%%%%%%%%%%%%%

  \vspace{-3.50 em}

  %%%%%%%%%%%%%%%%%%%%%%%%%%%%%%%%%%%%%%%%%%%%%%%%%%
  %% Skills section.
  \cvsection{\faCode}{Compétences techniques}

  \skillEntry{\faCoffee}{Écosystème \Java}{\SpringBoot, \RxJava, \Hibernate, \QueryDSL, \Flyway, \Testcontainers, \Gradle.}
  \skillEntry{\faDatabase}{BDD \& Système}{\Postgres, \MongoDB, \ElasticSearch, \Scylla, \Bash, \Docker, \Linux.}
  \skillEntry{\faGlobeAfrica}{Web \& Messaging}{\REST, \OpenAPI, \JWT, \OAuthTwo, \NATS, \PubSub, \React, \TypeScript.}
  \skillEntry{\faProjectDiagram}{Méthodologies}{\HexagonalArch, \CleanCode, \TDD, \Scrum, \Kanban, \DX.}
  \skillEntry{\faTools}{Outils}{\IntelliJ, \Git, \GitHub, \GitHubActions, \Postman, \Lens, \Grafana, \Jira, \LinkLaTeX.}

  %%%%%%%%%%%%%%%%%%%%%%%%%%%%%%%%%%%%%%%%%%%%%%%%%%

  \vspace{\cvSectionSpace}

  %%%%%%%%%%%%%%%%%%%%%%%%%%%%%%%%%%%%%%%%%%%%%%%%%%
  %% Experiences section.
  \cvsection{\faBriefcase}{Expériences professionnelles}

  \experienceEntry{\Happn}{Paris}{\remoteIcon}{Février 2024}{}{Java Software Engineer}{\#Rencontre \#App \#Legacy \faHeart}{%
    \experienceEntryContent{\happnURL}{./img/happn}
    {Développeur dans l'équipe backend d'\Happn, l'application de rencontre qui permet de retrouver les personnes croisées, que ce soit dans la rue, un café, ou les transports.}
    {
        {Développement de nouvelles features phares (\textit{\SmartLOL}, \textit{\PerfectDates}).},
        {Migration des microservices existants vers une \MakeLowercase{\HexagonalArch}.},
        {Dette \& upgrades: \Java{} 11/17$\;$\rightarrow{} 21, \SpringBoot{} 2$\;$\rightarrow{} 3, \Maven{}$\;$\rightarrow{} \Gradle.}
    }
    {\JavaTwentyOne, \SpringBoot, \ElasticSearch, \Scylla, \PubSub, \OpenAPI}
  }

  \experienceEntry{\SetKeeper}{Paris}{\remoteIcon}{Juin 2022}{Novembre 2023}{Java Software Engineer}{\#Cinéma \#Paie \#US \faFilm}{%
    \experienceEntryContent{\setkeeperURL}{./img/setkeeper}
    {Développeur backend dans l'équipe produit de \SetKeeper, une plateforme de gestion, d'organisation et de planification pour les équipes de tournage \href{\setkeeperClientsURL}{(films, séries, TV)}.}
    {
        {Mise en place d'outils et méthodes pour améliorer la \textit{Developer Experience} (\DX).},
        {Intégration de \Google{} (\GoogleDrive{} \& \GoogleContacts) pour du partage de documents générés.},
        {Migration d'une partie du code legacy vers une stack plus à jour.}
    }
    {\JavaSeventeen, \Vertx, \RxJava, \MongoDB, \Maven, \Docker, \OpenAPI}
  }

  \experienceEntry{\Metroscope}{Paris}{\remoteIcon}{Août 2019}{Juin 2022}{Full Stack Developer}{\#Energie \#Maintenance \#\IoT{} \faIndustry}{%
    \experienceEntryContent{\metroscopeURL}{./img/metroscope}
    {Développeur front puis back dans l'équipe produit de \Metroscope, un logiciel d'aide à l'analyse et la détection de défaillances sur les systèmes de production d'énergie.}
    {
        {Réécriture \textit{*from scratch*} de l'application web en \React{} et \TypeScript.},
        {Division du backend monolithique en microservices \& création de nouveaux services.},
        {Passage à une communication évènementielle entre les différents microservices.},
        {Veille et partage de connaissances via l'organisation de \textit{"Backend Chapters"}.}
    }
    {\JavaEleven, \Kotlin, \SpringBoot, \NATS, \Docker, \React, \TypeScript}
  }

  \experienceEntry{\Wemanity}{Paris}{}{Mars 2018}{Juillet 2019}{Full Stack Lead Developer}{\#Banque \#Innovation \#IA \faMoneyBill*}{%
    \experienceEntryContent{\wemanityURL}{./img/wemanity}
    {Lead Developer \RD{} au sein du Lab Innovation \textit{Innov8} de la \SG.}
    {
        {Idéation, création et maintenance d'applications autour des robots \Pepper.},
        {Développement d'un MVP de data-visualisation de graphes complexes en 3D.},
        {Étude et prototypage de services cognitifs (\OCR, \NLU, \STT{} et \Traduction).},
        {Formation et encadrement des developpeurs (stagiaires, alternants et juniors).}
    }
    {\React, \MaterialUI, \Loopback, \Python, \Docker, \GitLab}
  }

  \experienceEntry{\Dhatim}{Paris}{}{Octobre 2016}{Février 2018}{Java Developer}{\#Paie \#Comptabilité \#RH \faFileInvoiceDollar}{%
    \experienceEntryContent{\dhatimURL}{./img/dhatim}
    {Développeur backend dans l'équipe de \textit{\Conciliator}, une solution en SaaS de contrôle de \DSN{} pour les entreprises.}
    {
        {Prototypage et développement des nouvelles fonctionalités, \Scrum{} Master.},
        {Compréhension des besoins pour du support technique auprès des utilisateurs.},
        {Mise en place d'un POC de chatbot \Intercom{} pour du support client simple.}
    }
    {\JavaEight, \Dropwizard, \TestNG, \Postgres, \GitHub, \Jenkins}
  }

  \experienceEntry{\Visian}{Nanterre}{}{Mars 2015}{Octobre 2016}{\IoT{} Lead Developer}{\#\IoT{} \#\RD{} \#Innovation \faMicrochip}{%
    \experienceEntryContent{\visianURL}{./img/visian}
    {Développeur \RD{} puis Lead Developer chez \Visian, la filiale \IoT{} de \NeuronesIT.}
    {
        {Animation d'ateliers d'idéation, compréhension des besoins, veille technique.},
        {Développement de prototypes mêlant \IoT{}, mobile, cloud et data-visualisation.},
        {Encadrement et mentorat des équipes techniques (hardware \& software).}
    }
    {\LanguageCCPlusPlus, \JavaEight, \Android, \Python, \GitHub, \MicrosoftAzure}
  }

  %%%%%%%%%%%%%%%%%%%%%%%%%%%%%%%%%%%%%%%%%%%%%%%%%%

  \vspace{\cvSectionSpace}

  %%%%%%%%%%%%%%%%%%%%%%%%%%%%%%%%%%%%%%%%%%%%%%%%%%
  %% Education section.
  \cvsection{\faGraduationCap}{Parcours scolaire, stages \& alternances}

  \educationEntry{\Epitech}{2011}{2015}{\faUniversity}{Titre d'Expert en Technologie de l'Information}{Toulouse/Paris}{%
    \jobItems{
        {\companyName{\SII}: Prototypage \IoT{} combinant \href{\bitcrazeURL}{drone}, \RaspberryPi{} et divers capteurs (1 an).},
        {\companyName{\Thales}: Étude et déploiement d'\OpenStack{} dans un environnement de tests (6 mois).},
        {\companyName{\Novacom}: Développement d'un ensemble de tests unitaires et fonctionnels (1 an).}
    }
  }
  \educationEntry{\Griffith}{2013}{2014}{\faBeer}{Master I en Computing option Business \& Management}{Dublin}{}
  \educationEntry{\EISTI}{2008}{2011}{\faBrain}{Classes préparatoires Maths Sup/Spé (équivalence L2 Math/Info)}{Pau}{}
  \educationEntry{\Vauvenargues}{}{2008}{\faSchool}{Bac S options Maths \& Sciences de l'Ingénieur}{Aix-en-Provence}{}

  %%%%%%%%%%%%%%%%%%%%%%%%%%%%%%%%%%%%%%%%%%%%%%%%%%

  \vspace{\cvSectionSpace}

  %%%%%%%%%%%%%%%%%%%%%%%%%%%%%%%%%%%%%%%%%%%%%%%%%%
  %% Hobbies & Community sections.
  \begin{minipage}[t]{0.42\linewidth}
    \cvsection{\faTheaterMasks}{Centres d'intérêts}
    \extraEntry{\faPhotoVideo}{Photographie, musique, cinéma.}%
    \extraEntry{\faBook}{Lecture (thrillers, SF, fantasy, comics).}%
    \extraEntry{\faGamepad}{Lego, robotique, gaming, open source.}%
    \extraEntry{\faRunning}{\href{https://youtu.be/dQw4w9WgXcQ}{Urban fitting}.}%
  \end{minipage}
%
  \hspace{+2.00 cm}%
%
  \begin{minipage}[t]{0.42\linewidth}
    \cvsection{\faPeopleCarry}{Associatif}
    \extraEntry{\faCampground}{Scoutisme (en tant que scout et chef).}%
    \extraEntry{\faRobot}{Membre du Lab Robotique à \Epitech}%
    \extraEntry{\faCameraRetro}{Photographe du BDE de l'\EISTI.}%
    \extraEntry{\faVideo}{Responsable \textit{Cinéma \& Séries} à \AirEISTI.}%
  \end{minipage}

  %%%%%%%%%%%%%%%%%%%%%%%%%%%%%%%%%%%%%%%%%%%%%%%%%%

\end{document}


%% Personal information.
\name{\textbf{Bertrand}}{\textbf{Boyer}}
\title{\texorpdfstring{Java Software Engineer\newline\large{\SpringBoot{} - \RD{} - Innovation}}{Java Software Engineer}}
\address{}{}{}
\phone[mobile]{+33 (0) 6 10 04 27 16}
% \phone[fixed]{}
% \phone[fax]{}
\email{boyer.bertrand@gmail.com}
\social[linkedin]{BertrandBoyer}
\social[github]{BrtrndB}
\homepage{brtrndb.github.io}
% \social[twitter]{}
% \photo[64pt][0.5pt]{./img/profile}
% \extrainfo{\href{https://youtu.be/pq31rjX1BMw}{Epitech Promotion 2015}}
\quote{%
  \vspace{-2.50 em}\newline%
  « Basically, I transform bad coffee \faCoffee{} into good code \faKeyboard. »%
}%

\begin{document}
  %%%%%%%%%%%%%%%%%%%%%%%%%%%%%%%%%%%%%%%%%%%%%%%%%%
  %% Import custom macros for CV layout.
  % !TEX encoding = UTF-8 Unicode
% !TEX TS-program = LuaLaTeX

%% Packages.
\documentclass[10pt,a4paper]{moderncv}
\usepackage{luatextra}
\usepackage[french]{babel}
\usepackage{geometry}
\usepackage[fixed]{fontawesome5}
\usepackage{wrapfig}
\usepackage{ifthen}
\usepackage{xcolor}
\usepackage{textcomp}
\usepackage{eso-pic}
\usepackage{xargs}
\usepackage{expl3}
\usepackage{multirow}
\usepackage{adjustbox}

% Margin options.
\geometry{top=1.50 cm}
\geometry{bottom=1.50 cm}
\geometry{left=1.50 cm}
\geometry{right=1.50 cm}

%% moderncv configuration options.
\moderncvcolor{blue}
\moderncvstyle{classic}
\moderncvicons{awesome}
\nopagenumbers{}
\renewcommand{\familydefault}{\sfdefault}
\setlength{\hintscolumnwidth}{+3.20 cm} % Column width of cvitem.

%% Page border.
\newlength{\pageborder}
\setlength{\pageborder}{3 pt}
\pagecolor{color1}
\AddToShipoutPictureBG{%
  \AtPageLowerLeft{%
    \color{white}%
    \hspace{\pageborder}%
    \rule[\pageborder]{\paperwidth-2\pageborder}{\paperheight-2\pageborder}%
  }%
}

%% Import aliases.
\input{./BertrandBoyer.aliases}

%% Personal information.
\name{\textbf{Bertrand}}{\textbf{Boyer}}
\title{\texorpdfstring{Java Software Engineer\newline\large{\SpringBoot{} - \RD{} - Innovation}}{Java Software Engineer}}
\address{}{}{}
\phone[mobile]{+33 (0) 6 10 04 27 16}
% \phone[fixed]{}
% \phone[fax]{}
\email{boyer.bertrand@gmail.com}
\social[linkedin]{BertrandBoyer}
\social[github]{BrtrndB}
\homepage{brtrndb.github.io}
% \social[twitter]{}
% \photo[64pt][0.5pt]{./img/profile}
% \extrainfo{\href{https://youtu.be/pq31rjX1BMw}{Epitech Promotion 2015}}
\quote{%
  \vspace{-2.50 em}\newline%
  « Basically, I transform bad coffee \faCoffee{} into good code \faKeyboard. »%
}%

\begin{document}
  %%%%%%%%%%%%%%%%%%%%%%%%%%%%%%%%%%%%%%%%%%%%%%%%%%
  %% Import custom macros for CV layout.
  \input{./BertrandBoyer.macros}

  %%%%%%%%%%%%%%%%%%%%%%%%%%%%%%%%%%%%%%%%%%%%%%%%%%
  %% Title & Personal information.

  \makecvtitle

  %%%%%%%%%%%%%%%%%%%%%%%%%%%%%%%%%%%%%%%%%%%%%%%%%%

  \vspace{-3.50 em}

  %%%%%%%%%%%%%%%%%%%%%%%%%%%%%%%%%%%%%%%%%%%%%%%%%%
  %% Skills section.
  \cvsection{\faCode}{Compétences techniques}

  \skillEntry{\faCoffee}{Écosystème \Java}{\SpringBoot, \RxJava, \Hibernate, \QueryDSL, \Flyway, \Testcontainers, \Gradle.}
  \skillEntry{\faDatabase}{BDD \& Système}{\Postgres, \MongoDB, \ElasticSearch, \Scylla, \Bash, \Docker, \Linux.}
  \skillEntry{\faGlobeAfrica}{Web \& Messaging}{\REST, \OpenAPI, \JWT, \OAuthTwo, \NATS, \PubSub, \React, \TypeScript.}
  \skillEntry{\faProjectDiagram}{Méthodologies}{\HexagonalArch, \CleanCode, \TDD, \Scrum, \Kanban, \DX.}
  \skillEntry{\faTools}{Outils}{\IntelliJ, \Git, \GitHub, \GitHubActions, \Postman, \Lens, \Grafana, \Jira, \LinkLaTeX.}

  %%%%%%%%%%%%%%%%%%%%%%%%%%%%%%%%%%%%%%%%%%%%%%%%%%

  \vspace{\cvSectionSpace}

  %%%%%%%%%%%%%%%%%%%%%%%%%%%%%%%%%%%%%%%%%%%%%%%%%%
  %% Experiences section.
  \cvsection{\faBriefcase}{Expériences professionnelles}

  \experienceEntry{\Happn}{Paris}{\remoteIcon}{Février 2024}{}{Java Software Engineer}{\#Rencontre \#App \#Legacy \faHeart}{%
    \experienceEntryContent{\happnURL}{./img/happn}
    {Développeur dans l'équipe backend d'\Happn, l'application de rencontre qui permet de retrouver les personnes croisées, que ce soit dans la rue, un café, ou les transports.}
    {
        {Développement de nouvelles features phares (\textit{\SmartLOL}, \textit{\PerfectDates}).},
        {Migration des microservices existants vers une \MakeLowercase{\HexagonalArch}.},
        {Dette \& upgrades: \Java{} 11/17$\;$\rightarrow{} 21, \SpringBoot{} 2$\;$\rightarrow{} 3, \Maven{}$\;$\rightarrow{} \Gradle.}
    }
    {\JavaTwentyOne, \SpringBoot, \ElasticSearch, \Scylla, \PubSub, \OpenAPI}
  }

  \experienceEntry{\SetKeeper}{Paris}{\remoteIcon}{Juin 2022}{Novembre 2023}{Java Software Engineer}{\#Cinéma \#Paie \#US \faFilm}{%
    \experienceEntryContent{\setkeeperURL}{./img/setkeeper}
    {Développeur backend dans l'équipe produit de \SetKeeper, une plateforme de gestion, d'organisation et de planification pour les équipes de tournage \href{\setkeeperClientsURL}{(films, séries, TV)}.}
    {
        {Mise en place d'outils et méthodes pour améliorer la \textit{Developer Experience} (\DX).},
        {Intégration de \Google{} (\GoogleDrive{} \& \GoogleContacts) pour du partage de documents générés.},
        {Migration d'une partie du code legacy vers une stack plus à jour.}
    }
    {\JavaSeventeen, \Vertx, \RxJava, \MongoDB, \Maven, \Docker, \OpenAPI}
  }

  \experienceEntry{\Metroscope}{Paris}{\remoteIcon}{Août 2019}{Juin 2022}{Full Stack Developer}{\#Energie \#Maintenance \#\IoT{} \faIndustry}{%
    \experienceEntryContent{\metroscopeURL}{./img/metroscope}
    {Développeur front puis back dans l'équipe produit de \Metroscope, un logiciel d'aide à l'analyse et la détection de défaillances sur les systèmes de production d'énergie.}
    {
        {Réécriture \textit{*from scratch*} de l'application web en \React{} et \TypeScript.},
        {Division du backend monolithique en microservices \& création de nouveaux services.},
        {Passage à une communication évènementielle entre les différents microservices.},
        {Veille et partage de connaissances via l'organisation de \textit{"Backend Chapters"}.}
    }
    {\JavaEleven, \Kotlin, \SpringBoot, \NATS, \Docker, \React, \TypeScript}
  }

  \experienceEntry{\Wemanity}{Paris}{}{Mars 2018}{Juillet 2019}{Full Stack Lead Developer}{\#Banque \#Innovation \#IA \faMoneyBill*}{%
    \experienceEntryContent{\wemanityURL}{./img/wemanity}
    {Lead Developer \RD{} au sein du Lab Innovation \textit{Innov8} de la \SG.}
    {
        {Idéation, création et maintenance d'applications autour des robots \Pepper.},
        {Développement d'un MVP de data-visualisation de graphes complexes en 3D.},
        {Étude et prototypage de services cognitifs (\OCR, \NLU, \STT{} et \Traduction).},
        {Formation et encadrement des developpeurs (stagiaires, alternants et juniors).}
    }
    {\React, \MaterialUI, \Loopback, \Python, \Docker, \GitLab}
  }

  \experienceEntry{\Dhatim}{Paris}{}{Octobre 2016}{Février 2018}{Java Developer}{\#Paie \#Comptabilité \#RH \faFileInvoiceDollar}{%
    \experienceEntryContent{\dhatimURL}{./img/dhatim}
    {Développeur backend dans l'équipe de \textit{\Conciliator}, une solution en SaaS de contrôle de \DSN{} pour les entreprises.}
    {
        {Prototypage et développement des nouvelles fonctionalités, \Scrum{} Master.},
        {Compréhension des besoins pour du support technique auprès des utilisateurs.},
        {Mise en place d'un POC de chatbot \Intercom{} pour du support client simple.}
    }
    {\JavaEight, \Dropwizard, \TestNG, \Postgres, \GitHub, \Jenkins}
  }

  \experienceEntry{\Visian}{Nanterre}{}{Mars 2015}{Octobre 2016}{\IoT{} Lead Developer}{\#\IoT{} \#\RD{} \#Innovation \faMicrochip}{%
    \experienceEntryContent{\visianURL}{./img/visian}
    {Développeur \RD{} puis Lead Developer chez \Visian, la filiale \IoT{} de \NeuronesIT.}
    {
        {Animation d'ateliers d'idéation, compréhension des besoins, veille technique.},
        {Développement de prototypes mêlant \IoT{}, mobile, cloud et data-visualisation.},
        {Encadrement et mentorat des équipes techniques (hardware \& software).}
    }
    {\LanguageCCPlusPlus, \JavaEight, \Android, \Python, \GitHub, \MicrosoftAzure}
  }

  %%%%%%%%%%%%%%%%%%%%%%%%%%%%%%%%%%%%%%%%%%%%%%%%%%

  \vspace{\cvSectionSpace}

  %%%%%%%%%%%%%%%%%%%%%%%%%%%%%%%%%%%%%%%%%%%%%%%%%%
  %% Education section.
  \cvsection{\faGraduationCap}{Parcours scolaire, stages \& alternances}

  \educationEntry{\Epitech}{2011}{2015}{\faUniversity}{Titre d'Expert en Technologie de l'Information}{Toulouse/Paris}{%
    \jobItems{
        {\companyName{\SII}: Prototypage \IoT{} combinant \href{\bitcrazeURL}{drone}, \RaspberryPi{} et divers capteurs (1 an).},
        {\companyName{\Thales}: Étude et déploiement d'\OpenStack{} dans un environnement de tests (6 mois).},
        {\companyName{\Novacom}: Développement d'un ensemble de tests unitaires et fonctionnels (1 an).}
    }
  }
  \educationEntry{\Griffith}{2013}{2014}{\faBeer}{Master I en Computing option Business \& Management}{Dublin}{}
  \educationEntry{\EISTI}{2008}{2011}{\faBrain}{Classes préparatoires Maths Sup/Spé (équivalence L2 Math/Info)}{Pau}{}
  \educationEntry{\Vauvenargues}{}{2008}{\faSchool}{Bac S options Maths \& Sciences de l'Ingénieur}{Aix-en-Provence}{}

  %%%%%%%%%%%%%%%%%%%%%%%%%%%%%%%%%%%%%%%%%%%%%%%%%%

  \vspace{\cvSectionSpace}

  %%%%%%%%%%%%%%%%%%%%%%%%%%%%%%%%%%%%%%%%%%%%%%%%%%
  %% Hobbies & Community sections.
  \begin{minipage}[t]{0.42\linewidth}
    \cvsection{\faTheaterMasks}{Centres d'intérêts}
    \extraEntry{\faPhotoVideo}{Photographie, musique, cinéma.}%
    \extraEntry{\faBook}{Lecture (thrillers, SF, fantasy, comics).}%
    \extraEntry{\faGamepad}{Lego, robotique, gaming, open source.}%
    \extraEntry{\faRunning}{\href{https://youtu.be/dQw4w9WgXcQ}{Urban fitting}.}%
  \end{minipage}
%
  \hspace{+2.00 cm}%
%
  \begin{minipage}[t]{0.42\linewidth}
    \cvsection{\faPeopleCarry}{Associatif}
    \extraEntry{\faCampground}{Scoutisme (en tant que scout et chef).}%
    \extraEntry{\faRobot}{Membre du Lab Robotique à \Epitech}%
    \extraEntry{\faCameraRetro}{Photographe du BDE de l'\EISTI.}%
    \extraEntry{\faVideo}{Responsable \textit{Cinéma \& Séries} à \AirEISTI.}%
  \end{minipage}

  %%%%%%%%%%%%%%%%%%%%%%%%%%%%%%%%%%%%%%%%%%%%%%%%%%

\end{document}


  %%%%%%%%%%%%%%%%%%%%%%%%%%%%%%%%%%%%%%%%%%%%%%%%%%
  %% Title & Personal information.

  \makecvtitle

  %%%%%%%%%%%%%%%%%%%%%%%%%%%%%%%%%%%%%%%%%%%%%%%%%%

  \vspace{-3.50 em}

  %%%%%%%%%%%%%%%%%%%%%%%%%%%%%%%%%%%%%%%%%%%%%%%%%%
  %% Skills section.
  \cvsection{\faCode}{Compétences techniques}

  \skillEntry{\faCoffee}{Écosystème \Java}{\SpringBoot, \RxJava, \Hibernate, \QueryDSL, \Flyway, \Testcontainers, \Gradle.}
  \skillEntry{\faDatabase}{BDD \& Système}{\Postgres, \MongoDB, \ElasticSearch, \Scylla, \Bash, \Docker, \Linux.}
  \skillEntry{\faGlobeAfrica}{Web \& Messaging}{\REST, \OpenAPI, \JWT, \OAuthTwo, \NATS, \PubSub, \React, \TypeScript.}
  \skillEntry{\faProjectDiagram}{Méthodologies}{\HexagonalArch, \CleanCode, \TDD, \Scrum, \Kanban, \DX.}
  \skillEntry{\faTools}{Outils}{\IntelliJ, \Git, \GitHub, \GitHubActions, \Postman, \Lens, \Grafana, \Jira, \LinkLaTeX.}

  %%%%%%%%%%%%%%%%%%%%%%%%%%%%%%%%%%%%%%%%%%%%%%%%%%

  \vspace{\cvSectionSpace}

  %%%%%%%%%%%%%%%%%%%%%%%%%%%%%%%%%%%%%%%%%%%%%%%%%%
  %% Experiences section.
  \cvsection{\faBriefcase}{Expériences professionnelles}

  \experienceEntry{\Happn}{Paris}{\remoteIcon}{Février 2024}{}{Java Software Engineer}{\#Rencontre \#App \#Legacy \faHeart}{%
    \experienceEntryContent{\happnURL}{./img/happn}
    {Développeur dans l'équipe backend d'\Happn, l'application de rencontre qui permet de retrouver les personnes croisées, que ce soit dans la rue, un café, ou les transports.}
    {
        {Développement de nouvelles features phares (\textit{\SmartLOL}, \textit{\PerfectDates}).},
        {Migration des microservices existants vers une \MakeLowercase{\HexagonalArch}.},
        {Dette \& upgrades: \Java{} 11/17$\;$\rightarrow{} 21, \SpringBoot{} 2$\;$\rightarrow{} 3, \Maven{}$\;$\rightarrow{} \Gradle.}
    }
    {\JavaTwentyOne, \SpringBoot, \ElasticSearch, \Scylla, \PubSub, \OpenAPI}
  }

  \experienceEntry{\SetKeeper}{Paris}{\remoteIcon}{Juin 2022}{Novembre 2023}{Java Software Engineer}{\#Cinéma \#Paie \#US \faFilm}{%
    \experienceEntryContent{\setkeeperURL}{./img/setkeeper}
    {Développeur backend dans l'équipe produit de \SetKeeper, une plateforme de gestion, d'organisation et de planification pour les équipes de tournage \href{\setkeeperClientsURL}{(films, séries, TV)}.}
    {
        {Mise en place d'outils et méthodes pour améliorer la \textit{Developer Experience} (\DX).},
        {Intégration de \Google{} (\GoogleDrive{} \& \GoogleContacts) pour du partage de documents générés.},
        {Migration d'une partie du code legacy vers une stack plus à jour.}
    }
    {\JavaSeventeen, \Vertx, \RxJava, \MongoDB, \Maven, \Docker, \OpenAPI}
  }

  \experienceEntry{\Metroscope}{Paris}{\remoteIcon}{Août 2019}{Juin 2022}{Full Stack Developer}{\#Energie \#Maintenance \#\IoT{} \faIndustry}{%
    \experienceEntryContent{\metroscopeURL}{./img/metroscope}
    {Développeur front puis back dans l'équipe produit de \Metroscope, un logiciel d'aide à l'analyse et la détection de défaillances sur les systèmes de production d'énergie.}
    {
        {Réécriture \textit{*from scratch*} de l'application web en \React{} et \TypeScript.},
        {Division du backend monolithique en microservices \& création de nouveaux services.},
        {Passage à une communication évènementielle entre les différents microservices.},
        {Veille et partage de connaissances via l'organisation de \textit{"Backend Chapters"}.}
    }
    {\JavaEleven, \Kotlin, \SpringBoot, \NATS, \Docker, \React, \TypeScript}
  }

  \experienceEntry{\Wemanity}{Paris}{}{Mars 2018}{Juillet 2019}{Full Stack Lead Developer}{\#Banque \#Innovation \#IA \faMoneyBill*}{%
    \experienceEntryContent{\wemanityURL}{./img/wemanity}
    {Lead Developer \RD{} au sein du Lab Innovation \textit{Innov8} de la \SG.}
    {
        {Idéation, création et maintenance d'applications autour des robots \Pepper.},
        {Développement d'un MVP de data-visualisation de graphes complexes en 3D.},
        {Étude et prototypage de services cognitifs (\OCR, \NLU, \STT{} et \Traduction).},
        {Formation et encadrement des developpeurs (stagiaires, alternants et juniors).}
    }
    {\React, \MaterialUI, \Loopback, \Python, \Docker, \GitLab}
  }

  \experienceEntry{\Dhatim}{Paris}{}{Octobre 2016}{Février 2018}{Java Developer}{\#Paie \#Comptabilité \#RH \faFileInvoiceDollar}{%
    \experienceEntryContent{\dhatimURL}{./img/dhatim}
    {Développeur backend dans l'équipe de \textit{\Conciliator}, une solution en SaaS de contrôle de \DSN{} pour les entreprises.}
    {
        {Prototypage et développement des nouvelles fonctionalités, \Scrum{} Master.},
        {Compréhension des besoins pour du support technique auprès des utilisateurs.},
        {Mise en place d'un POC de chatbot \Intercom{} pour du support client simple.}
    }
    {\JavaEight, \Dropwizard, \TestNG, \Postgres, \GitHub, \Jenkins}
  }

  \experienceEntry{\Visian}{Nanterre}{}{Mars 2015}{Octobre 2016}{\IoT{} Lead Developer}{\#\IoT{} \#\RD{} \#Innovation \faMicrochip}{%
    \experienceEntryContent{\visianURL}{./img/visian}
    {Développeur \RD{} puis Lead Developer chez \Visian, la filiale \IoT{} de \NeuronesIT.}
    {
        {Animation d'ateliers d'idéation, compréhension des besoins, veille technique.},
        {Développement de prototypes mêlant \IoT{}, mobile, cloud et data-visualisation.},
        {Encadrement et mentorat des équipes techniques (hardware \& software).}
    }
    {\LanguageCCPlusPlus, \JavaEight, \Android, \Python, \GitHub, \MicrosoftAzure}
  }

  %%%%%%%%%%%%%%%%%%%%%%%%%%%%%%%%%%%%%%%%%%%%%%%%%%

  \vspace{\cvSectionSpace}

  %%%%%%%%%%%%%%%%%%%%%%%%%%%%%%%%%%%%%%%%%%%%%%%%%%
  %% Education section.
  \cvsection{\faGraduationCap}{Parcours scolaire, stages \& alternances}

  \educationEntry{\Epitech}{2011}{2015}{\faUniversity}{Titre d'Expert en Technologie de l'Information}{Toulouse/Paris}{%
    \jobItems{
        {\companyName{\SII}: Prototypage \IoT{} combinant \href{\bitcrazeURL}{drone}, \RaspberryPi{} et divers capteurs (1 an).},
        {\companyName{\Thales}: Étude et déploiement d'\OpenStack{} dans un environnement de tests (6 mois).},
        {\companyName{\Novacom}: Développement d'un ensemble de tests unitaires et fonctionnels (1 an).}
    }
  }
  \educationEntry{\Griffith}{2013}{2014}{\faBeer}{Master I en Computing option Business \& Management}{Dublin}{}
  \educationEntry{\EISTI}{2008}{2011}{\faBrain}{Classes préparatoires Maths Sup/Spé (équivalence L2 Math/Info)}{Pau}{}
  \educationEntry{\Vauvenargues}{}{2008}{\faSchool}{Bac S options Maths \& Sciences de l'Ingénieur}{Aix-en-Provence}{}

  %%%%%%%%%%%%%%%%%%%%%%%%%%%%%%%%%%%%%%%%%%%%%%%%%%

  \vspace{\cvSectionSpace}

  %%%%%%%%%%%%%%%%%%%%%%%%%%%%%%%%%%%%%%%%%%%%%%%%%%
  %% Hobbies & Community sections.
  \begin{minipage}[t]{0.42\linewidth}
    \cvsection{\faTheaterMasks}{Centres d'intérêts}
    \extraEntry{\faPhotoVideo}{Photographie, musique, cinéma.}%
    \extraEntry{\faBook}{Lecture (thrillers, SF, fantasy, comics).}%
    \extraEntry{\faGamepad}{Lego, robotique, gaming, open source.}%
    \extraEntry{\faRunning}{\href{https://youtu.be/dQw4w9WgXcQ}{Urban fitting}.}%
  \end{minipage}
%
  \hspace{+2.00 cm}%
%
  \begin{minipage}[t]{0.42\linewidth}
    \cvsection{\faPeopleCarry}{Associatif}
    \extraEntry{\faCampground}{Scoutisme (en tant que scout et chef).}%
    \extraEntry{\faRobot}{Membre du Lab Robotique à \Epitech}%
    \extraEntry{\faCameraRetro}{Photographe du BDE de l'\EISTI.}%
    \extraEntry{\faVideo}{Responsable \textit{Cinéma \& Séries} à \AirEISTI.}%
  \end{minipage}

  %%%%%%%%%%%%%%%%%%%%%%%%%%%%%%%%%%%%%%%%%%%%%%%%%%

\end{document}


  %%%%%%%%%%%%%%%%%%%%%%%%%%%%%%%%%%%%%%%%%%%%%%%%%%
  %% Title & Personal information.

  \makecvtitle

  %%%%%%%%%%%%%%%%%%%%%%%%%%%%%%%%%%%%%%%%%%%%%%%%%%

  \vspace{-3.50 em}

  %%%%%%%%%%%%%%%%%%%%%%%%%%%%%%%%%%%%%%%%%%%%%%%%%%
  %% Skills section.
  \cvsection{\faCode}{Compétences techniques}

  \skillEntry{\faCoffee}{Écosystème \Java}{\SpringBoot, \RxJava, \Hibernate, \QueryDSL, \Flyway, \Testcontainers, \Gradle.}
  \skillEntry{\faDatabase}{BDD \& Système}{\Postgres, \MongoDB, \ElasticSearch, \Scylla, \Bash, \Docker, \Linux.}
  \skillEntry{\faGlobeAfrica}{Web \& Messaging}{\REST, \OpenAPI, \JWT, \OAuthTwo, \NATS, \PubSub, \React, \TypeScript.}
  \skillEntry{\faProjectDiagram}{Méthodologies}{\HexagonalArch, \CleanCode, \TDD, \Scrum, \Kanban, \DX.}
  \skillEntry{\faTools}{Outils}{\IntelliJ, \Git, \GitHub, \GitHubActions, \Postman, \Lens, \Grafana, \Jira, \LinkLaTeX.}

  %%%%%%%%%%%%%%%%%%%%%%%%%%%%%%%%%%%%%%%%%%%%%%%%%%

  \vspace{\cvSectionSpace}

  %%%%%%%%%%%%%%%%%%%%%%%%%%%%%%%%%%%%%%%%%%%%%%%%%%
  %% Experiences section.
  \cvsection{\faBriefcase}{Expériences professionnelles}

  \experienceEntry{\Happn}{Paris}{\remoteIcon}{Février 2024}{}{Java Software Engineer}{\#Rencontre \#App \#Legacy \faHeart}{%
    \experienceEntryContent{\happnURL}{./img/happn}
    {Développeur dans l'équipe backend d'\Happn, l'application de rencontre qui permet de retrouver les personnes croisées, que ce soit dans la rue, un café, ou les transports.}
    {
        {Développement de nouvelles features phares (\textit{\SmartLOL}, \textit{\PerfectDates}).},
        {Migration des microservices existants vers une \MakeLowercase{\HexagonalArch}.},
        {Dette \& upgrades: \Java{} 11/17$\;$\rightarrow{} 21, \SpringBoot{} 2$\;$\rightarrow{} 3, \Maven{}$\;$\rightarrow{} \Gradle.}
    }
    {\JavaTwentyOne, \SpringBoot, \ElasticSearch, \Scylla, \PubSub, \OpenAPI}
  }

  \experienceEntry{\SetKeeper}{Paris}{\remoteIcon}{Juin 2022}{Novembre 2023}{Java Software Engineer}{\#Cinéma \#Paie \#US \faFilm}{%
    \experienceEntryContent{\setkeeperURL}{./img/setkeeper}
    {Développeur backend dans l'équipe produit de \SetKeeper, une plateforme de gestion, d'organisation et de planification pour les équipes de tournage \href{\setkeeperClientsURL}{(films, séries, TV)}.}
    {
        {Mise en place d'outils et méthodes pour améliorer la \textit{Developer Experience} (\DX).},
        {Intégration de \Google{} (\GoogleDrive{} \& \GoogleContacts) pour du partage de documents générés.},
        {Migration d'une partie du code legacy vers une stack plus à jour.}
    }
    {\JavaSeventeen, \Vertx, \RxJava, \MongoDB, \Maven, \Docker, \OpenAPI}
  }

  \experienceEntry{\Metroscope}{Paris}{\remoteIcon}{Août 2019}{Juin 2022}{Full Stack Developer}{\#Energie \#Maintenance \#\IoT{} \faIndustry}{%
    \experienceEntryContent{\metroscopeURL}{./img/metroscope}
    {Développeur front puis back dans l'équipe produit de \Metroscope, un logiciel d'aide à l'analyse et la détection de défaillances sur les systèmes de production d'énergie.}
    {
        {Réécriture \textit{*from scratch*} de l'application web en \React{} et \TypeScript.},
        {Division du backend monolithique en microservices \& création de nouveaux services.},
        {Passage à une communication évènementielle entre les différents microservices.},
        {Veille et partage de connaissances via l'organisation de \textit{"Backend Chapters"}.}
    }
    {\JavaEleven, \Kotlin, \SpringBoot, \NATS, \Docker, \React, \TypeScript}
  }

  \experienceEntry{\Wemanity}{Paris}{}{Mars 2018}{Juillet 2019}{Full Stack Lead Developer}{\#Banque \#Innovation \#IA \faMoneyBill*}{%
    \experienceEntryContent{\wemanityURL}{./img/wemanity}
    {Lead Developer \RD{} au sein du Lab Innovation \textit{Innov8} de la \SG.}
    {
        {Idéation, création et maintenance d'applications autour des robots \Pepper.},
        {Développement d'un MVP de data-visualisation de graphes complexes en 3D.},
        {Étude et prototypage de services cognitifs (\OCR, \NLU, \STT{} et \Traduction).},
        {Formation et encadrement des developpeurs (stagiaires, alternants et juniors).}
    }
    {\React, \MaterialUI, \Loopback, \Python, \Docker, \GitLab}
  }

  \experienceEntry{\Dhatim}{Paris}{}{Octobre 2016}{Février 2018}{Java Developer}{\#Paie \#Comptabilité \#RH \faFileInvoiceDollar}{%
    \experienceEntryContent{\dhatimURL}{./img/dhatim}
    {Développeur backend dans l'équipe de \textit{\Conciliator}, une solution en SaaS de contrôle de \DSN{} pour les entreprises.}
    {
        {Prototypage et développement des nouvelles fonctionalités, \Scrum{} Master.},
        {Compréhension des besoins pour du support technique auprès des utilisateurs.},
        {Mise en place d'un POC de chatbot \Intercom{} pour du support client simple.}
    }
    {\JavaEight, \Dropwizard, \TestNG, \Postgres, \GitHub, \Jenkins}
  }

  \experienceEntry{\Visian}{Nanterre}{}{Mars 2015}{Octobre 2016}{\IoT{} Lead Developer}{\#\IoT{} \#\RD{} \#Innovation \faMicrochip}{%
    \experienceEntryContent{\visianURL}{./img/visian}
    {Développeur \RD{} puis Lead Developer chez \Visian, la filiale \IoT{} de \NeuronesIT.}
    {
        {Animation d'ateliers d'idéation, compréhension des besoins, veille technique.},
        {Développement de prototypes mêlant \IoT{}, mobile, cloud et data-visualisation.},
        {Encadrement et mentorat des équipes techniques (hardware \& software).}
    }
    {\LanguageCCPlusPlus, \JavaEight, \Android, \Python, \GitHub, \MicrosoftAzure}
  }

  %%%%%%%%%%%%%%%%%%%%%%%%%%%%%%%%%%%%%%%%%%%%%%%%%%

  \vspace{\cvSectionSpace}

  %%%%%%%%%%%%%%%%%%%%%%%%%%%%%%%%%%%%%%%%%%%%%%%%%%
  %% Education section.
  \cvsection{\faGraduationCap}{Parcours scolaire, stages \& alternances}

  \educationEntry{\Epitech}{2011}{2015}{\faUniversity}{Titre d'Expert en Technologie de l'Information}{Toulouse/Paris}{%
    \jobItems{
        {\companyName{\SII}: Prototypage \IoT{} combinant \href{\bitcrazeURL}{drone}, \RaspberryPi{} et divers capteurs (1 an).},
        {\companyName{\Thales}: Étude et déploiement d'\OpenStack{} dans un environnement de tests (6 mois).},
        {\companyName{\Novacom}: Développement d'un ensemble de tests unitaires et fonctionnels (1 an).}
    }
  }
  \educationEntry{\Griffith}{2013}{2014}{\faBeer}{Master I en Computing option Business \& Management}{Dublin}{}
  \educationEntry{\EISTI}{2008}{2011}{\faBrain}{Classes préparatoires Maths Sup/Spé (équivalence L2 Math/Info)}{Pau}{}
  \educationEntry{\Vauvenargues}{}{2008}{\faSchool}{Bac S options Maths \& Sciences de l'Ingénieur}{Aix-en-Provence}{}

  %%%%%%%%%%%%%%%%%%%%%%%%%%%%%%%%%%%%%%%%%%%%%%%%%%

  \vspace{\cvSectionSpace}

  %%%%%%%%%%%%%%%%%%%%%%%%%%%%%%%%%%%%%%%%%%%%%%%%%%
  %% Hobbies & Community sections.
  \begin{minipage}[t]{0.42\linewidth}
    \cvsection{\faTheaterMasks}{Centres d'intérêts}
    \extraEntry{\faPhotoVideo}{Photographie, musique, cinéma.}%
    \extraEntry{\faBook}{Lecture (thrillers, SF, fantasy, comics).}%
    \extraEntry{\faGamepad}{Lego, robotique, gaming, open source.}%
    \extraEntry{\faRunning}{\href{https://youtu.be/dQw4w9WgXcQ}{Urban fitting}.}%
  \end{minipage}
%
  \hspace{+2.00 cm}%
%
  \begin{minipage}[t]{0.42\linewidth}
    \cvsection{\faPeopleCarry}{Associatif}
    \extraEntry{\faCampground}{Scoutisme (en tant que scout et chef).}%
    \extraEntry{\faRobot}{Membre du Lab Robotique à \Epitech}%
    \extraEntry{\faCameraRetro}{Photographe du BDE de l'\EISTI.}%
    \extraEntry{\faVideo}{Responsable \textit{Cinéma \& Séries} à \AirEISTI.}%
  \end{minipage}

  %%%%%%%%%%%%%%%%%%%%%%%%%%%%%%%%%%%%%%%%%%%%%%%%%%

\end{document}


  %%%%%%%%%%%%%%%%%%%%%%%%%%%%%%%%%%%%%%%%%%%%%%%%%%
  %% Title & Personal information.

  \makecvtitle

  %%%%%%%%%%%%%%%%%%%%%%%%%%%%%%%%%%%%%%%%%%%%%%%%%%

  \vspace{-3.50 em}

  %%%%%%%%%%%%%%%%%%%%%%%%%%%%%%%%%%%%%%%%%%%%%%%%%%
  %% Skills section.
  \cvsection{\faCode}{Compétences techniques}

  \skillEntry{\faCoffee}{Écosystème \Java}{\SpringBoot, \RxJava, \Hibernate, \QueryDSL, \Flyway, \Testcontainers, \Gradle.}
  \skillEntry{\faDatabase}{BDD \& Système}{\Postgres, \MongoDB, \ElasticSearch, \Scylla, \Bash, \Docker, \Linux.}
  \skillEntry{\faGlobeAfrica}{Web \& Messaging}{\REST, \OpenAPI, \JWT, \OAuthTwo, \NATS, \PubSub, \React, \TypeScript.}
  \skillEntry{\faProjectDiagram}{Méthodologies}{\HexagonalArch, \CleanCode, \TDD, \Scrum, \Kanban, \DX.}
  \skillEntry{\faTools}{Outils}{\IntelliJ, \Git, \GitHub, \GitHubActions, \Postman, \Lens, \Grafana, \Jira, \LinkLaTeX.}

  %%%%%%%%%%%%%%%%%%%%%%%%%%%%%%%%%%%%%%%%%%%%%%%%%%

  \vspace{\cvSectionSpace}

  %%%%%%%%%%%%%%%%%%%%%%%%%%%%%%%%%%%%%%%%%%%%%%%%%%
  %% Experiences section.
  \cvsection{\faBriefcase}{Expériences professionnelles}

  \experienceEntry{\Happn}{Paris}{\remoteIcon}{Février 2024}{}{Java Software Engineer}{\#Rencontre \#App \#Legacy \faHeart}{%
    \experienceEntryContent{\happnURL}{./img/happn}
    {Développeur dans l'équipe backend d'\Happn, l'application de rencontre qui permet de retrouver les personnes croisées, que ce soit dans la rue, un café, ou les transports.}
    {
        {Développement de nouvelles features phares (\textit{\SmartLOL}, \textit{\PerfectDates}).},
        {Migration des microservices existants vers une \MakeLowercase{\HexagonalArch}.},
        {Passage de \Maven{} à \Gradle{} en mutualisant la configuration via l'écriture de plugins.},
        {Regroupement des librairies et microservices les plus à jour dans un unique repo.},
        {Réduction de la dette technique: \Java{} 8/11/17 vers 21, \SpringBoot{} 2 vers 3.}
    }
    {\JavaTwentyOne, \SpringBoot, \ElasticSearch, \Scylla, \PubSub, \OpenAPI}
  }

  \experienceEntry{\SetKeeper}{Paris}{\remoteIcon}{Juin 2022}{Novembre 2023}{Java Software Engineer}{\#Cinéma \#Paie \#US \faFilm}{%
    \experienceEntryContent{\setkeeperURL}{./img/setkeeper}
    {Développeur backend dans l'équipe produit de \SetKeeper, une plateforme de gestion, d'organisation et de planification pour les équipes de tournage \href{\setkeeperClientsURL}{(films, séries, TV)}.}
    {
        {Mise en place d'outils et méthodes pour améliorer la \textit{Developer Experience} (\DX).},
        {Écriture d'un ensemble de scénarios de tests d'intégration avec \Testcontainers.},
        {Intégration de \Google{} (\GoogleContacts{} \& \GoogleDrive) pour du partage de documents.},
        {Réécriture et migration d'une partie du code legacy vers une stack plus récente.}
    }
    {\JavaSeventeen, \Vertx, \RxJava, \MongoDB, \Maven, \Docker, \OpenAPI}
  }

  \experienceEntry{\Metroscope}{Paris}{\remoteIcon}{Août 2019}{Juin 2022}{Full Stack Developer}{\#Energie \#Maintenance \#\IoT{} \faIndustry}{%
    \experienceEntryContent{\metroscopeURL}{./img/metroscope}
    {Développeur front puis back dans l'équipe produit de \Metroscope, un logiciel d'aide à l'analyse et la détection de défaillances sur les systèmes de production d'énergie.}
    {
        {Réécriture \textit{*from scratch*} de l'application web en \React{} et \TypeScript.},
        {Division du backend monolithique en microservices et repos indépendants.},
        {Passage à une communication évènementielle entre les différents microservices.},
        {Regroupement de tous les repositories du backend au sein d'un mono-repo.},
        {Veille et partage de connaissances via l'organisation de \textit{"Backend Chapters"}.}
    }
    {\JavaEleven, \Kotlin, \SpringBoot, \NATS, \Docker, \React, \TypeScript}
  }

  \experienceEntry{\Wemanity}{Paris}{}{Mars 2018}{Juillet 2019}{Full Stack Lead Developer}{\#Banque \#Innovation \#IA \faMoneyBill*}{%
    \experienceEntryContent{\wemanityURL}{./img/wemanity}
    {Lead Developer \RD{} au sein du Lab Innovation \textit{Innov8} de la \SG.}
    {
        {Idéation, création et maintenance d'applications autour des robots \Pepper.},
        {Développement d'un MVP de data-visualisation de graphes complexes en 3D.},
        {Étude et prototypage de services cognitifs (\OCR, \NLU, \STT{} et \Traduction).},
        {Formation et encadrement des developpeurs (stagiaires, alternants et juniors).}
    }
    {\React, \MaterialUI, \ThreeJS, \Loopback, \Python, \Docker, \GitLab}
  }

  \experienceEntry{\Dhatim}{Paris}{}{Octobre 2016}{Février 2018}{Java Developer}{\#Paie \#Comptabilité \#RH \faFileInvoiceDollar}{%
    \experienceEntryContent{\dhatimURL}{./img/dhatim}
    {Développeur backend dans l'équipe de \textit{\Conciliator}, une solution en SaaS de contrôle de \DSN{} pour les entreprises.}
    {
        {Prototypage et développement des nouvelles fonctionalités, \Scrum{} Master.},
        {Compréhension des besoins pour du support technique auprès des utilisateurs.},
        {Mise en place d'un POC de chatbot \Intercom{} pour du support client simple.}
    }
    {\JavaEight, \Dropwizard, \TestNG, \Postgres, \GitHub, \Jenkins}
  }

  \experienceEntry{\Visian}{Nanterre}{}{Mars 2015}{Octobre 2016}{\IoT{} Lead Developer}{\#\IoT{} \#\RD{} \#Innovation \faMicrochip}{%
    \experienceEntryContent{\visianURL}{./img/visian}
    {Développeur \RD{} puis Lead Developer chez \Visian, la filiale \IoT{} de \NeuronesIT.}
    {
        {Animation d'ateliers d'idéation, compréhension des besoins, veille technique.},
        {Développement de prototypes mêlant \IoT{}, mobile, cloud et data-visualisation.},
        {Encadrement et mentorat des équipes techniques (hardware \& software).}
    }
    {\LanguageCCPlusPlus, \JavaEight, \Android, \Python, \GitHub, \MicrosoftAzure}
  }

  %%%%%%%%%%%%%%%%%%%%%%%%%%%%%%%%%%%%%%%%%%%%%%%%%%

  \vspace{\cvSectionSpace}

  %%%%%%%%%%%%%%%%%%%%%%%%%%%%%%%%%%%%%%%%%%%%%%%%%%
  %% Education section.
  \cvsection{\faGraduationCap}{Parcours académique, stages \& alternances}

  \educationEntry{\Epitech}{2011}{2015}{\faUniversity}{Titre d'Expert en Technologie de l'Information}{Toulouse/Paris}{%
    \jobItems{
        {\companyName{\SII}: Prototypage \IoT{} combinant \href{\bitcrazeURL}{drone}, \RaspberryPi{} et divers capteurs (1 an).},
        {\companyName{\Thales}: Étude et déploiement d'\OpenStack{} dans un environnement de tests (6 mois).},
        {\companyName{\Novacom}: Développement d'un ensemble de tests unitaires et fonctionnels (1 an).}
    }
  }
  \educationEntry{\Griffith}{2013}{2014}{\faBeer}{Master I en Computing option Business \& Management}{Dublin}{}
  \educationEntry{\EISTI}{2008}{2011}{\faBrain}{Classes préparatoires Maths Sup/Spé (équivalence L2 Math/Info)}{Pau}{}
  \educationEntry{\Vauvenargues}{}{2008}{\faSchool}{Bac S options Maths \& Sciences de l'Ingénieur}{Aix-en-Provence}{}

  %%%%%%%%%%%%%%%%%%%%%%%%%%%%%%%%%%%%%%%%%%%%%%%%%%

  \vspace{\cvSectionSpace}

  %%%%%%%%%%%%%%%%%%%%%%%%%%%%%%%%%%%%%%%%%%%%%%%%%%
  %% Hobbies & Community sections.
  \par\addvspace{+0.50 em}
  \begin{minipage}[t]{0.42\linewidth}
    \cvsection{\faTheaterMasks}{Centres d'intérêts}
    \extraEntry{\faPhotoVideo}{Photographie, musique, cinéma.}%
    \extraEntry{\faBook}{Lecture (thrillers, SF, fantasy, comics).}%
    \extraEntry{\faGamepad}{Lego, jeux video, cuisine, open source.}%
    \extraEntry{\faBicycle}{Balades en vélo, \href{https://youtu.be/dQw4w9WgXcQ}{urban fitting}.}%
  \end{minipage}
%
  \hspace{+2.00 cm}%
%
  \begin{minipage}[t]{0.42\linewidth}
    \cvsection{\faPeopleCarry}{Associatif}
    \extraEntry{\faCampground}{Scoutisme (en tant que scout et chef).}%
    \extraEntry{\faRobot}{Membre du Lab Robotique à \Epitech}%
    \extraEntry{\faCameraRetro}{Photographe du BDE de l'\EISTI.}%
    \extraEntry{\faVideo}{Responsable \textit{Cinéma \& Séries} à \AirEISTI.}%
  \end{minipage}

  %%%%%%%%%%%%%%%%%%%%%%%%%%%%%%%%%%%%%%%%%%%%%%%%%%

\end{document}
