%% Lengths.
\newlength{\jobEntrySpace}       %% Vertical space between each experience entry.
\newlength{\jobLogoWidth}        %% Vertical space between each experience entry.
\setlength{\jobEntrySpace}{+0.50 em}
\setlength{\jobLogoWidth}{0.20\maincolumnwidth}

%% Macros.
\newcommand{\jobRemote}{\faHouseUser}
\newcommand{\jobNoRemote}{\faBuilding}
\newcommand{\yearMONTH}[1]{\scriptsize{\getNth{#1}{ }{1}}\hspace{+0.20 em}\small{\getNth{#1}{ }{2}}}
\newcommandx{\jobInfo}[4][4={\jobNoRemote},usedefault]{%
    \setlength{\fboxsep}{0pt}
    \adjustbox{valign=t}{%
        \begin{tabular}{rc}
            \IfBlankTF{#2}{
                \yearMONTH{#1}                      & \small{\faCalendarDay}\\[-3pt]
            }%
            {%
                \yearMONTH{#1}                      & \multirow{2}{*}{\small{\faCalendarDay}}\\[-3pt]
                \yearMONTH{#2}                      & \\[-3pt]
            }%
            \IfBlankF{#3}{%
                \scriptsize{\textcolor{color2}{#3}} & \scriptsize{\textcolor{color2}{#4}}\\[-3pt]
            }%
        \end{tabular}%
    }%
}
\newcommand{\jobCompany}[1]{\textcolor{color1}{#1}}
\newcommand{\jobTitle}[1]{#1}
\newcommand{\jobTags}[1]{\normalfont{\scriptsize{\textcolor{color2}{#1}}}\hspace{\jobLogoWidth}}
\newcommandx{\jobEntry}[8][]{
    \cventry[\jobEntrySpace]
    {\jobInfo{#4}{#5}{#2}[#3]}
    {\jobCompany{#1}\IfBlankT{#6}{\IfBlankF{#7}{\hfill\jobTags{#7}}}}
    {\IfBlankF{#6}{\jobTitle{#6}\IfBlankF{#7}{\hfill\jobTags{#7}}}}
    {}
    {}
    {#8}
}
\newcommandx{\jobEntryContent}[5][]{%
    \IfBlankTF{#2}{%
        \parbox{0.80\maincolumnwidth}{%
            \IfBlankF{#3}{\jobDescription{#3}\newline}
            \IfBlankF{#4}{\jobItems{#4}\newline}
            \IfBlankF{#5}{\jobSkills{#5}}
        }
    }
    {%
        \begin{tabular}{c@{\hspace{1.00 em}}c}
            \parbox{0.80\maincolumnwidth}{%
                \IfBlankF{#3}{\jobDescription{#3}\newline}
                \IfBlankF{#4}{\jobItems{#4}}
                \IfBlankF{#5}{\jobSkills{#5}}
            } &
            \parbox{\jobLogoWidth}{%
                \href{#1}{\includegraphics[width=\jobLogoWidth]{#2}}} \\
        \end{tabular}%
    }
}
\newcommand{\jobDescription}[1]{#1}
\newcommand{\jobItems}[1]{
    \foreach \x in {#1}{%
        \textcolor{color1}{\hspace{+0.20 em}\faAngleRight}\x\newline%
    }%
}
\newcommand{\jobSkills}[1]{\textcolor{color2}{Stack technique: #1.}}


%% Experiences section.
\cvsection{\faBriefcase}{Expériences professionnelles}

\jobEntry{\Happn}{Paris}{\jobRemote}{Février 2024}{}{Java Software Engineer}{\#Dating \#App \#Legacy \faHeart}{%
    \jobEntryContent{\happnURL}{./img/experiences/happn}
    {Développeur dans l'équipe backend d'\Happn, l'app de rencontre qui permet de retrouver les personnes croisées, que ce soit dans la rue, un café, ou les transports.}
    {
            {Développement de nouvelles features (\textit{\SmartLOL}, \textit{\PerfectDates}).},
            {Migration des microservices existants en architecture hexagonale.},
            {Dette \& upgrades: \Java{} 11/17$\;$\rightarrow{} 21, \SpringBoot{} 2$\;$\rightarrow{} 3, \Maven{}$\;$\rightarrow{} \Gradle.}
    }
    {\JavaTwentyOne, \SpringBoot, \ElasticSearch, \Scylla, \PubSub, \GCP}
}

\jobEntry{\SetKeeper}{Paris}{\jobRemote}{Juin 2022}{Novembre 2023}{Java Software Engineer}{\#Cinéma \#Paie \#US \faFilm}{%
    \jobEntryContent{\setkeeperURL}{./img/experiences/setkeeper}
    {Développeur backend dans l'équipe produit de \SetKeeper, une plateforme de gestion, d'organisation et de planification pour les équipes de tournage \href{\setkeeperClientsURL}{(films, séries, TV)}.}
    {
            {Migration d'une partie du code legacy sur une stack plus à jour.},
            {Mise en place d'outils et méthodes pour améliorer la \textit{Developer Experience}~(\DX).},
            {Intégration avec \Google{}, via \OAuthTwo{}, et de ses APIs (\GoogleDrive, \GoogleContacts).}
    }
    {\JavaSeventeen, \Vertx, \MongoDB, \Maven, \Docker, \OpenAPI}
}

\jobEntry{\Metroscope}{Paris}{\jobRemote}{Août 2019}{Juin 2022}{Full Stack Developer}{\#Industrie \#Monitoring \#\IoT{} \faIndustry}{%
    \jobEntryContent{\metroscopeURL}{./img/experiences/metroscope}
    {Développeur front puis back dans l'équipe produit de \Metroscope, un logiciel d'aide à l'analyse et la détection de défaillances sur les systèmes de production d'énergie.}
    {
            {Réécriture \textit{*from scratch*} de l'application web en \React{} et \TypeScript.},
            {Division du backend en microservices et dévelopement des nouveaux services.},
            {Passage d'une communication synchrone à asynchrone entre les services.},
            {Veille et partage de connaissances via l'organisation de \textit{"Backend Chapters"}.}
    }
    {\JavaEleven, \Kotlin, \SpringBoot, \NATS, \Docker, \React, \TypeScript}
}

\jobEntry{\Wemanity}{Paris}{}{Mars 2018}{Juillet 2019}{Full Stack Lead Developer}{\#Banque \#Innovation \#IA \faMoneyBill*}{%
    \jobEntryContent{\wemanityURL}{./img/experiences/wemanity}
    {Lead Developer \RD{} au sein du Lab Innovation \textit{Innov8} de la \SG.}
    {
            {Idéation, création et maintenance d'applications autour des robots \Pepper.},
            {Développement d'un MVP de data-visualisation en 3D de graphes complexes.},
            {Étude et prototypage de services cognitifs (\OCR, \NLU, \STT{} et \Traduction).},
            {Formation et encadrement des developpeurs (stagiaires, alternants et juniors).}
    }
    {\React, \MaterialUI, \Loopback, \Python, \Docker, \GitLab}
}

\jobEntry{\Dhatim}{Paris}{}{Octobre 2016}{Février 2018}{Java Developer}{\#Paie \#Comptabilité \#RH \faFileInvoiceDollar}{%
    \jobEntryContent{\dhatimURL}{./img/experiences/dhatim}
    {Développeur backend dans l'équipe de \textit{\Conciliator}, une solution en SaaS de contrôle de \DSN{} pour les entreprises.}
    {
            {Compréhension des besoins métiers et support utilisateur sur site.},
            {Développement des nouvelles fonctionalités, \Scrum{} Master.},
            {Prototypage d'un chatbot \Intercom{} pour du support client simple.}
    }
    {\JavaEight, \Dropwizard, \TestNG, \Postgres, \GitHub, \Jenkins}
}

\jobEntry{\Visian}{Nanterre}{}{Mars 2015}{Octobre 2016}{\IoT{} Lead Developer}{\#\IoT{} \#\RD{} \#Innovation \faMicrochip}{%
    \jobEntryContent{\visianURL}{./img/experiences/visian}
    {Développeur \RD{} puis Lead Developer chez \Visian, la filiale \IoT{} de \NeuronesIT.}
    {
            {Animation d'ateliers d'idéation, compréhension des besoins, veille technique.},
            {Développement de prototypes mêlant \IoT{}, mobile, cloud et data-visualisation.},
            {Encadrement et mentorat des équipes techniques (hardware \& software).}
    }
    {\LanguageCCPlusPlus, \JavaEight, \Android, \Python, \GitHub, \MicrosoftAzure}
}

\jobEntry{Stages \& alternances}{Paris/Toulouse}{}{Avril 2013}{Mars 2015}{}{\#\IoT{} \#DevOps \#Tests \faUserGraduate}{%
    \jobEntryContent{}{}{}
    {
            {\jobCompany{\SII}: Prototypage \IoT{} combinant \href{\bitcrazeURL}{drone}, \RaspberryPi{} et divers capteurs.},
            {\jobCompany{\Thales}: Étude et déploiement d'\OpenStack{} dans un environnement de tests.},
            {\jobCompany{\Novacom}: Développement d'un ensemble de tests unitaires et fonctionnels.}
    }
    {}
}
