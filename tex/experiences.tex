%% Lengths.
\newlength{\cvExperienceEntrySpace}       %% Vertical space between each experience entry.
\setlength{\cvExperienceEntrySpace}{+0.25 cm}
\newlength{\logoColumnWidth}              %% Vertical space between each experience entry.
\setlength{\logoColumnWidth}{0.18\textwidth}

%% Macros.
\newcommand{\jobdate}[2]{
    \IfBlankTF{#2}
    {#1\hspace{+0.50 em}\faCalendarDay}
    {#1\hspace{+0.50 em}\rightharpoonup\\#2\hspace{+0.50 em}\faCalendarDay}
}
\newcommand{\jobtags}[1]{\normalfont{\scriptsize{\textcolor{color2}{#1}}}\hspace{\logoColumnWidth}}
\newcommand{\joblogo}[4]{%
    \displayYes{\displayImageExperiences}{%
        \vspace{-0.20 cm}
        \begin{wrapfigure}{R}{0.2\textwidth}%
        \IfBlankF{#3}{\vspace{#3}}%
        \centering%
        \href{#1}{\includegraphics[width=\logoColumnWidth]{#2}}%
        \IfBlankF{#4}{\vspace{#4}}%
        \end{wrapfigure}%
    }
}
\newcommand{\jobtitle}[1]{#1}
\newcommand{\jobplace}[1]{\textcolor{color1}{#1}}
\newcommand{\joblocation}[1]{\textcolor{color2}{#1\hspace{+0.50 em}\faMapMarker*}}
\newcommand{\jobdescription}[1]{#1}
\newcommand{\jobitems}[1]{
    \foreach \x in {#1}{%
        \textcolor{color1}{\hspace{+1.00 em}\faAngleRight} \x\newline%
    }%
}
\newcommand{\jobstack}[1]{\textcolor{color2}{Stack technique: #1.}}

\newcommand{\cvExperienceEntry}[7]{
    \cventry[\cvExperienceEntrySpace]
    {\jobdate{#3}{#4}\IfBlankF{#2}{\\\joblocation{#2}}}%
    {\jobplace{#1}\IfBlankT{#5}{\IfBlankF{#6}{\hfill\jobtags{#6}}}}
    {\IfBlankF{#5}{\jobtitle{#5}\IfBlankF{#6}{\hfill\jobtags{#6}}}}
    {}
    {}
    {#7}
}

%% Experiences section.
\cvsection{\faBriefcase}{Expériences professionnelles}

\cvExperienceEntry{\Happn}{Paris}{Février 2024}{}{Java Software Engineer}{\#Dating \#App \#Legacy \faHeart}{%
    \joblogo{\happnURL}{./img/experiences/happn}{-0.50 cm}{}%
    \jobdescription{Développeur dans l'équipe backend d'\Happn, l'app de rencontre qui permet de retrouver les personnes croisées, que ce soit dans la rue, un café, ou les transports.}\newline%
    \jobitems{
            {Développement de nouvelles features (\textit{\SmartLOL}, \textit{\PerfectDates}).},
            {Migration des microservices existants en architecture hexagonale.},
            {Dette \& upgrades: \Java{} 11/17$\;$\rightarrow{} 21, \SpringBoot{} 2$\;$\rightarrow{} 3, \Maven{}$\;$\rightarrow{} \Gradle.}
    }
    \jobstack{\JavaTwentyOne, \SpringBoot, \ElasticSearch, \Scylla, \PubSub, \GCP}%
}

\cvExperienceEntry{\SetKeeper}{Paris}{Juin 2022}{Novembre 2023}{Java Software Engineer}{\#Cinéma \#Paie \#US \faFilm}{%
    \joblogo{\setkeeperURL}{./img/experiences/setkeeper}{-0.30 cm}{}%
    \jobdescription{Développeur backend dans l'équipe produit de \SetKeeper, une plateforme de gestion, d'organisation et de planification pour les équipes de tournage \href{\setkeeperClientsURL}{(films, séries, TV)}.}\newline%
    \jobitems{
            {Migration d'une partie du code legacy sur une stack plus à jour.},
            {Mise en place d'outils et méthodes pour améliorer la \textit{Developer Experience}~(\DX).},
            {Intégration avec \Google{}, via \OAuthTwo{}, et de ses APIs (\GoogleDrive, \GoogleContacts).}
    }
    \jobstack{\JavaSeventeen, \Vertx, \MongoDB, \Maven, \Docker, \OpenAPI}%
}

\cvExperienceEntry{\Metroscope}{Paris}{Août 2019}{Juin 2022}{Full Stack Developer}{\#Industrie \#Monitoring \#\IoT{} \faIndustry}{%
    \joblogo{\metroscopeURL}{./img/experiences/metroscope}{-0.40 cm}{}%
    \jobdescription{Développeur front puis back dans l'équipe produit de \Metroscope, un logiciel d'aide à l'analyse et la détection de défaillances sur les systèmes de production d'énergie.}\newline%
    \jobitems{
            {Réécriture \textit{*from scratch*} de l'application web en \React{} et \TypeScript.},
            {Division du backend en microservices et dévelopement des nouveaux services.},
            {Passage d'une communication synchrone à asynchrone entre les services.},
            {Veille et partage de connaissances via l'organisation de \textit{"Backend Chapters"}.}
    }
    \jobstack{\JavaEleven, \Kotlin, \SpringBoot, \NATS, \Docker, \React, \TypeScript}%
}

\cvExperienceEntry{\Wemanity}{Paris}{Mars 2018}{Juillet 2019}{Full Stack Lead Developer}{\#Banque \#Innovation \#IA \faMoneyBill*}{%
    \joblogo{\wemanityURL}{./img/experiences/wemanity}{-0.30 cm}{}%
    \jobdescription{Lead Developer \RD{} au sein du Lab Innovation \textit{Innov8} de la \SG.}\newline%
    \jobitems{
            {Idéation, création et maintenance d'applications autour des robots \Pepper.},
            {Développement d'un MVP de data-visualisation en 3D de graphes complexes.},
            {Étude et prototypage de services cognitifs (\OCR, \NLU, \STT{} et \Traduction).},
            {Formation et encadrement des developpeurs (stagiaires, alternants et juniors).}
    }
    \jobstack{\React, \MaterialUI, \Loopback, \Python, \Docker, \GitLab}%
}

\cvExperienceEntry{\Dhatim}{Paris}{Octobre 2016}{Février 2018}{Java Developer}{\#Paie \#Comptabilité \#RH \faFileInvoiceDollar}{%
    \joblogo{\dhatimURL}{./img/experiences/dhatim}{}{}%
    \jobdescription{Développeur backend dans l'équipe de \textit{\Conciliator}, une solution en SaaS de contrôle de \DSN{} pour les entreprises.}\newline%
    \jobitems{
            {Compréhension des besoins métiers et support utilisateur sur site.},
            {Développement des nouvelles fonctionalités, \Scrum{} Master.},
            {Prototypage d'un chatbot \Intercom{} pour du support client simple.}
    }
    \jobstack{\JavaEight, \Dropwizard, \TestNG, \Postgres, \GitHub, \Jenkins}%
}

\cvExperienceEntry{\Visian}{Nanterre}{Mars 2015}{Octobre 2016}{\IoT{} Lead Developer}{\#\IoT{} \#\RD{} \#Innovation \faMicrochip}{%
    \joblogo{\visianURL}{./img/experiences/visian}{}{}%
    \jobdescription{Développeur \RD{} puis Lead Developer chez \Visian, la filiale \IoT{} de \NeuronesIT.}\newline%
    \jobitems{
            {Animation d'ateliers d'idéation, compréhension des besoins, veille technique.},
            {Développement de prototypes mêlant \IoT{}, mobile, cloud et data-visualisation.},
            {Encadrement et mentorat des équipes techniques (hardware \& software).}
    }
    \jobstack{\LanguageCCPlusPlus, \JavaEight, \Android, \Python, \GitHub, \MicrosoftAzure}%
}

\cvExperienceEntry{Stages \& alternances}{}{Avril 2013}{Mars 2015}{}{\#\IoT{} \#DevOps \#Tests \faUserGraduate}{%
    \jobitems{
            {\jobplace{\SII}: Prototypage \IoT{} combinant \href{\bitcrazeURL}{drone}, \RaspberryPi{} et divers capteurs.},
            {\jobplace{\Thales}: Étude et déploiement d'\OpenStack{} dans un environnement de tests.},
            {\jobplace{\Novacom}: Développement d'un ensemble de tests unitaires et fonctionnels.}
    }
}
